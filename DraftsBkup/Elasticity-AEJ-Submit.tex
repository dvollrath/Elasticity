\documentclass[11pt]{article}
%%%%%%%%%%%%%%%%%%%%%%%%%%%%%%%%%%%%%%%%
\usepackage{amsmath}
\usepackage{verbatim}
\usepackage[usenames,dvipsnames]{color}
\usepackage{setspace}
\usepackage{lscape}
\usepackage{longtable}
\usepackage[top=1.25in,bottom=1.25in,left=1in,right=1in]{geometry}
\usepackage{graphicx}
\usepackage{epstopdf}
\usepackage{epsfig}
\usepackage{fancyhdr}
\usepackage{booktabs}
\usepackage{dcolumn}
\usepackage{arydshln}
\usepackage{natbib}
\usepackage{tabularx}
\usepackage{subfigure}
\usepackage{hyperref}
\usepackage{xcolor}

{\newcommand{\baseearlydepr}{0.19}
{\newcommand{\baseearlynoprofit}{0.32}
{\newcommand{\baselatedepr}{0.24}
{\newcommand{\baselatenoprofit}{0.37}
{\newcommand{\excldepr}{0.11}
{\newcommand{\exclmaxdepr}{0.14}
{\newcommand{\exclnoprofit}{0.26}
{\newcommand{\exclmaxnoprofit}{0.30}
{\newcommand{\baseinv}{0.26}
{\newcommand{\noipearlydepr}{0.16}
{\newcommand{\noipearlynoprofit}{0.29}
{\newcommand{\noiplatedepr}{0.19}
{\newcommand{\noiplatenoprofit}{0.33}
{\newcommand{\nohsearlydepr}{0.13}
{\newcommand{\nohsearlynoprofit}{0.27}
{\newcommand{\nohslatedepr}{0.17}
{\newcommand{\nohslatenoprofit}{0.31}
{\newcommand{\basestdepr}{0.09}
{\newcommand{\basetnoprofit}{0.16}
{\newcommand{\baseeqdepr}{0.08}
{\newcommand{\baseeqnoprofit}{0.13}
{\newcommand{\baseipdepr}{0.03}
{\newcommand{\baseipnoprofit}{0.05}
{\newcommand{\baseipearlydepr}{0.01}
{\newcommand{\baseipearlynoprofit}{0.03}
{\newcommand{\baseiplatedepr}{0.06}
{\newcommand{\baseiplatenoprofit}{0.08}
{\newcommand{\basefirstnoprofit}{0.33}
{\newcommand{\baselastnoprofit}{0.39}
{\newcommand{\basefirstdepr}{0.16}
{\newcommand{\baselastdepr}{0.25}
{\newcommand{\basediffdepr}{0.10}
{\newcommand{\basefirstinv}{0.22}
{\newcommand{\baselastinv}{0.29}
{\newcommand{\basemednoprofit}{   .}
{\newcommand{\basemeddepr}{   .}
{\newcommand{\basemeannoprofit}{0.337}
{\newcommand{\baseminnoprofit}{0.291}
{\newcommand{\basemaxnoprofit}{0.389}
 % commands for counts of observations

\hypersetup{
    colorlinks,
    linkcolor={red!50!black},
    citecolor={blue!50!black},
    urlcolor={blue!80!black}
}

\newtheorem{proposition}{Proposition}
\newtheorem{corollary}{Corollary}

\setcounter{MaxMatrixCols}{10}
\newcolumntype{d}[1]{D{.}{.}{-2.#1}}
\newenvironment{proof}[1][Proof]{\noindent\textbf{#1.} }{\ \rule{0.5em}{0.5em}}
\setlength{\columnsep}{.2in}
%\psset{unit=1cm}
\newcolumntype{R}{>{\raggedleft\arraybackslash}X}

\def\sym#1{\ifmmode^{#1}\else\(^{#1}\)\fi}

\begin{document}
\begin{titlepage}
\vspace{2in} \noindent {\large \today}

\vspace{.5in} \noindent {\Large \textbf{\strut The Elasticity of Aggregate Output with Respect to Capital and Labor}}

\vspace{.25in} \noindent {\large Dietrich Vollrath}

\vspace{.05in} \noindent University of Houston

\vfill \noindent \textsc{Abstract} \hrulefill

\vspace{.05in} \noindent It is often assumed that the elasticity of GDP with respect to capital is one-third, but this assumes zero markups and an aggregate production function. I estimate the elasticity allowing markups to vary by industry and with a rich input-output structure. Assumptions about capital costs provide bounds on the elasticity. In the U.S. from 1948-1995 the capital elasticity was in the range \baseearlydepr-\baseearlynoprofit \ and this shifted to \baselatedepr-\baselatenoprofit \ by 1996-2018. Excluding housing or de-capitalizing intellectual property lowers those bounds to as low as \excldepr-\exclnoprofit. Based on these elasticities, common estimates of total factor productivity growth represent a lower bound. 

\vspace{.1in} \hrule

\vspace{.5in} \noindent {\small JEL Codes: E01, E23, E25, E22, E13, N12, O4}

\vspace{.1in} \noindent {\small Keywords: capital elasticity, labor elasticity, labor share, national income accounting}

\vspace{.1in} \noindent {\small Contact information: 201C McElhinney Hall, U. of Houston, Houston, TX 77204, devollrath@uh.edu. I'd like to thank Eirini Magames for excellent research assistance. I also thank David Baqaee, David Papell, and Yona Rubinstein for comments on this project, as well as seminar participants at Iowa State, Clark University, Southern Methodist, and Texas Tech. All errors remain my own.}
\end{titlepage}

\pagebreak 

\section{Introduction}
\onehalfspacing One of the most common assumptions made within economics is that ``alpha equals one-third'', referring to the capital elasticity $\alpha$ in a Cobb-Douglas aggregate production function $Y = K^{\alpha}L^{1-\alpha}$. This rule of thumb is derived from an observation that labor's share of GDP is around two-thirds, implying $1-\alpha \approx 2/3$, and hence that $\alpha \approx 1/3$. Not only has a recent literature (reviewed below) documented that labor's share of GDP has fallen in the last few decades, but as \cite{hall1988,hall1990} notes this rule of thumb only works if one assumes there are zero economic profits and labor's GDP share is equal to it's elasticity. The rule of thumb relies on the existence of an aggregate production function and commonly assumes that $\alpha$ is constant over time.

In recent work \cite{bfshortnote,bfprodge} show that one can calculate a meaningful elasticity of GDP with respect to aggregate capital, $\partial \ln Y/ \partial \ln K$, without having to rely on any of the assumptions embedded in the rule of thumb. In particular, their theory shows how to calculate the aggregate elasticity from disaggregated units (e.g. industries) with rich input/output relationships and arbitrary unit-level distortions (e.g. markups). There is no need to assume an aggregate production function exists, that profits are zero, or that the elasticity remains constant over time. Their structure requires only market-clearing and cost-minimization. The same structure provides an elasticity with respect to labor, $\partial \ln Y/ \partial \ln L$.

In this paper I apply the Baqaee and Farhi theory to calculate the annual elasticity of GDP with respect to capital and labor in the United States from 1948-2018 using dis-aggregated data on industries and input-output relationships, allowing for arbitrary markups at the industry level.

The theory in \cite{bfshortnote,bfprodge} does not eliminate the well-known problem of separating capital costs from economic profits in national accounts data. Because of this, what I present here are plausible bounds on the capital and labor elasticities based on different assumptions regarding capital costs. An upper bound for the capital elasticity is established by assuming zero economic profits in all industries, such that all value-added not used for labor compensation is paid to capital, as in the rule of thumb. This represents one extreme of the capital elasticity, and true values are likely lower. A lower bound for the capital elasticity is found by assuming capital costs are equal to depreciation, as industries pay at least this amount in capital costs.\footnote{There are measurement issues with labor costs as well, in particular with the treatment of proprietors income \citep{Gollin:2002zr,gommerupert2004,elsbyhs2013}. In practice the treatment of proprietors income generates little variation in the estimated elasticities.} Bounds for the labor elasticity are one minus the capital elasticity, so the no-profit assumption represents a lower bound, and the depreciation assumption an upper bound, for the labor elasticity. 

My baseline bounds for the capital elasticity in the U.S. can be seen in Figure \ref{FIG_cap_comparison}. Between 1948 and 1995, the elasticity of GDP with respect to capital was in a range of \baseearlydepr-\baseearlynoprofit, with 1/3 forming a rough upper bound. After 1995 the range shifted up, and from 1996-2018 the elasticity with respect to capital was \baselatedepr-\baselatenoprofit. The bounds for the labor elasticity are the mirror image of these.

Because of the issues with measuring capital costs I cannot give a precise estimate of the capital elasticity or labor elasticity. To the extent that there are markups present in the economy, the actual capital elasticity must lie below the upper bound (and hence the labor elasticity above its lower bound). Given recent evidence that markups were above one throughout the period 1948-2018 \citep{Barkai000,edmondetal2018,NBERw23687,RePEc:nbr:nberwo:22897,basu2019} this implies that the capital elasticity was \textit{below} one-third from 1948-1995. After 1995, it becomes less clear. Higher markups in that period imply the capital elasticity was below the upper bound, but at the same time the upper bound on the capital elasticity rose. A value of one-third for the capital elasticity after 1995 is plausible, but in no way certain.

In addition to the bounding estimates, I explore two alternative ways of imputing the capital costs that go into the elasticity calculations. The first alternative is a user cost formula \citep{halljorg1967} as in \cite{Barkai000} and \cite{rognlie2015}. The elasticity estimates based on user costs of capital fluctuate from 1948-2018 and show an upward drift, but for the most part lie within the bounds. There are exceptions that imply periods of widespread negative economic profits.\footnote{User cost based estimates of the capital elasticity in 1975-1989 often lie above the upper bound, implying negative economic profits. This is driven by assumptions made regarding expected inflation in the user cost formula. I discuss this in Section 4 when covering the results in more detail.} As a second alternative, I use investment spending by industries to estimate their capital costs. The capital elasticity based on these costs shows a smaller upward trend, and stays everywhere within the bounds. On average, the capital elasticity is around \baseinv \ using investment to measures capital costs.

The bounds and alternative estimates of the capital elasticity all show an upward trend to some extent. I use an \cite{op1996} decomposition of the elasticity estimates to investigate the source of this. On average, a higher capital elasticity across all industries drove the trend in the bounds, and changes in the composition of GDP across industries did not. There is evidence that larger industries tend to have higher capital elasticities, but that relationship did not strengthen or weaken over time.

Beyond these baseline results, the bounds on the elasticities depend on the scope of economic activity included. In particular, if I narrow my focus to the private business sector (excluding government and owner-occupied housing) then the estimated capital elasticity is lower than in the baseline. The capital elasticity in the private business sector is \nohsearlydepr-\nohsearlynoprofit \ from 1948-1995, and \nohslatedepr-\nohslatenoprofit \ from 1996-2018, always below one-third. 

In a different exercise, I break down the aggregate capital elasticity by three types of capital: structures, equipment, and intellectual property. The elasticity with respect to structures is in the range \basestdepr-\basetnoprofit \ throughout the time period studied, and equipment in the range \baseeqdepr-\baseeqnoprofit. Intellectual property prior to 1960 has a low elasticity of \baseipearlydepr-\baseipearlynoprofit, but after 2000 it lies in the range \baseiplatedepr-\baseiplatenoprofit. Much of the apparent increase in the boundaries of the aggregate capital elasticity can be accounted for by this increase in the elasticity with respect to intellectual property.

As another way of assessing the importance of intellectual property, I de-capitalize it from the national accounts data as in \cite{ksz2020} and recompute the aggregate capital elasticity. From 1948-1995 the capital elasticity is in the range \noipearlydepr-\noipearlynoprofit, and \noiplatedepr-\noiplatenoprofit \ from 1996-2018. The combination of results suggests that IP accounts for much of the overall increase in the capital elasticity bounds over time.\footnote{Theoretically it would be possible to go in the other direction as well, and consider the elasticity estimates after \textit{capitalizing} other intermediate spending from the input/output tables (e.g. technical consulting services, engineering services, etc.) that might plausibly be thought to generate intangible capital \citep{chs2009,mp2010,MCGRATTAN2020S147}. The industry-level data available on an annual basis does not have enough detail to separate this spending out. Nevertheless, it would be correct to say that the capital elasticity I estimate in this paper is the elasticity of output with respect to \textit{measured} capital.} 

This paper is a complement to the growing literature on the distribution of GDP across factors.\footnote{\cite{azmat2012,bentolilaSP2003,estrada2014,harrison2005,jt2007,guscina2006,kn2014,daoetal2017} all document a decline in labor's share of GDP in the last few decades, across countries and industries. This was contemporaneous with a decline in capital's share of GDP \citep{Barkai000,rognlie2015}. Incorporating the lessons in \cite{Gollin:2002zr} regarding proprietors income does not appear to change that conclusion \citep{gommerupert2004,elsbyhs2013}. The decline in labor's share has been tied to a fall in the price of new capital \citep{kn2014}, but more recent research suggests it may be an artifact of capitalizing intellectual property \citep{ksz2020} or the reporting of income \citep{syzz2019}.} It shares with that literature the same measurement issues surrounding proprietors income and capital costs. Recent work on ``factorless income'' by \cite{kn2019} is perhaps the closest methodological analogue to this paper, in that those authors explore a range of plausible approaches for dealing with this factorless income at an aggregate level. The bounds I find for the elasticities are calculated either assuming that all factorless income is attributed to capital (the zero profit bound) or that all factorless income represents economic profits (the depreciation cost bound).\footnote{Factorless income as a share of value-added is larger in the industry-level data than in the aggregate because I do not have information on some rental costs that are reported at the aggregate level.} 

Despite the similarities in approach with the literature on factor shares of GDP, my findings are distinct. Elasticities and factor shares are not identical due both to market power and the input-output structure of the economy \citep{bfshortnote,bfprodge}. For this reason a factor's share of GDP does not necessarily inform us about the elasticity of GDP with respect to that factor.

One of the uses of the elasticities I estimate is as an input into other macroeconomic studies. Research on how the supply of factors of production (e.g. savings/consumption decisions, labor force participation, demographic change, international finance) affect the economy could use these elasticities without necessarily having to specify an entire production structure that incorporates market power or rich input-output relationships. As \cite{bfshortnote,bfprodge} show, these elasticities already embed those features, taking industry-level markups and total factor productivity as given. As I can only provide reasonable bounds for the elasticities, this provides a range of values that factor supply models could use to evaluate their results. One caution is that the elasticities are first-order approximations, and any dramatic changes in factor supplies would have to account for second-order effects \citep{bfmicro}. Further, with markups and productivity levels held constant, these are partial elasticities with respect to factor supplies, and do not encompass effects of endogenous reallocation or productivity change that might occur in response to changes in those factor supplies. 

A different application is to growth accounting, which depends on elasticity estimates to calculate growth in total factor productivity (TFP). Typical accounting exercises by the Bureau of Labor Statistics, as well as extensions to incorporate utilization rates \citep{kfb2006,fernald2014}, use the factor share of labor to find the labor elasticity and one minus that share to find the capital elasticity, which is equivalent to the no-profits bound on the elasticities. That likely overstates the capital elasticity and understates the labor elasticity. In the last section of this paper I use the bounds on the elasticities to create bounds for the growth of TFP in the U.S. from 1948-2018. The typical no-profit assumption represents a \textit{lower} bound for the growth rate of TFP, which averages about 1.29\% per year 1948-2018. The growth rate of TFP may have been up to 1.60\% per year, depending on the choice of assumption used to find the capital and labor elasticities. The level of TFP in 2018 may be up to 25\% higher than what the typical calculation finds. 

The paper proceeds as follows. Section 2 presents the theoretical framework of \cite{bfshortnote,bfprodge} I use to calculate the elasticities, and Section 3 discusses the data sources and major measurement issues. In Section 4 I present the baseline results on the bounds for the elasticities, as well as the alternatives based on investment and user cost assumptions. Section 5 evaluates trends in the elasticity estimates and their relationship to aggregate ratios of costs to GDP. In Section 6 I explore how the elasticities change depending on the scope of economic activity. Section 7 performs the growth accounting exercises, and Section 8 concludes.

\section{Theoretical background}
What I present in this section is a simplified version of \cite{bfshortnote,bfprodge} to highlight only the parts of their theory that I use. Full proofs and deeper explanations can be found in their papers. In the interest of space I use the abbreviation ``BF'' to refer to those authors in this section.

The economy consists of $J$ industries, and each industry uses intermediate inputs from other industries (and possibly itself), as well as the factors of production capital ($K$) and labor ($L$). There can be any arbitrary number of factors of production, and I use capital and labor here only for simplicity. BF's theory is also ``nest-able'' in that each industry could have an arbitrary number of sub-industries or firms inside of it. I am speaking of industries here solely because this is the level of detail I have available in the data. 

The gross production function of any industry has constant returns with respect to intermediates and the factors of production, but no other structure is imposed. Each industry is assumed to be cost-minimizing, and each industry charges a price for their output that is a markup over the marginal cost. For my purposes I will not need to know the price or the markup. It will be sufficient to speak only about the costs faced by each industry.

Everything I present in this section holds for a given period $t$. To avoid needless notation, for the remainder of this section I will not use the $t$ subscript. 

To begin, for industry $i$ let the costs of intermediate good purchased from industry $j$ be denoted as $COST_{ij}$. The sum of costs accounted for by intermediate goods purchased by industry $i$ from all other $J$ industries are then

\begin{equation}
	COST_{iM} = \sum_{j=1}^{J} COST_{ij}
\end{equation}

where the letter $M$ is used to denote that this represents intermediate good costs only.

The capital costs faced by industry $i$ will be denoted $COST_{iK}$, and the labor costs of the same industry will be denoted $COST_{iL}$. Combined with the intermediate good summation above, this means that total costs for industry $i$ are

\begin{equation}
	COST_i = COST_{iM} + COST_{iK} + COST_{iL}. \label{EQ_costi}
\end{equation}

Using these total costs one can calculate cost shares, which will be the most relevant piece of information for calculating the aggregate elasticities in the end. For industry $i$ the share of total costs accounted for by intermediate purchases from industry $j$ is defined as

\begin{equation}
	\lambda_{ij} = \frac{COST_{ij}}{COST_i}. \label{EQ_lambdaij}
\end{equation}

In a similar manner, for industry $i$ the share of total costs accounted for by capital and labor, respectively, are

\begin{eqnarray}
	\lambda_{iK} &=& \frac{COST_{iK}}{COST_i} \\
	\lambda_{iL} &=& \frac{COST_{iL}}{COST_i}.
\end{eqnarray}

These cost shares can be used to build a modified input-output matrix that BF show can be used to calculate the aggregate elasticities with respect to labor and capital. The insight from BF is that one can treat the factors of production as ``industries'' that serve as an input to other industries, but which purchase no intermediates from other industries. By including them in an expanded I/O matrix they show how this can be used to solve for the elasticity of aggregate output with respect to those factors. 

Let $\Lambda$ be a $J+2$ by $J+2$ matrix, which has $J$ rows/columns from the individual industries, and two additional rows/columns, one each for the capital and labor ``industries''. Each row of $\Lambda$ is associated with an industry $i$, and the entries show the cost share of industry $i$ coming from the industry $j$ represented in the columns. The cost shares of capital and labor for industry $i$ are included as the final two columns of each row. 

The last two \textit{rows} of $\Lambda$ capture the cost structure of the capital and labor, and they are ``dummy'' rows in the sense that each entry is a zero. The capital and labor industries do not employ any intermediate goods themselves, nor do they hire labor or capital.

It is easiest to understand the structure of $\Lambda$ by examining it,

\begin{equation}
	\Lambda = 
	\begin{bmatrix}
		\lambda_{11} & \lambda_{12} & \cdots & \lambda_{1J} & \lambda_{1K} & \lambda_{1L} \\
		\lambda_{21} & \lambda_{22} & \cdots & \lambda_{2J} & \lambda_{2K} & \lambda_{2L} \\
		\vdots       & \vdots       & \ddots & \vdots       & \vdots       & \vdots  \\
		\lambda_{J1} & \lambda_{J2} & \cdots & \lambda_{JJ} & \lambda_{JK} & \lambda_{JL} \\
		0 & 0 & \cdots & 0 & 0& 0 \\
		0 & 0 & \cdots & 0 & 0& 0 \\
	\end{bmatrix} \label{EQ_Lambda}
\end{equation} 

The top-left $JxJ$ block of this matrix are cost shares of intermediates in total costs. The final two columns represent the capital and labor cost shares, respectively, for each industry. Note that the sum across a row is equal to one for each of the first $J$ rows, simply indicating that the matrix accounts for the total costs facing industry $i$. The final two rows of the matrix are the dummy rows for capital (second to last) and labor (last row), and they sum to zero.

A calculation of aggregate elasticities requires one final piece of information. Let $f_j$ be the final use of output from industry $j$, and $FINAL = \sum_{j=1}^J f_j$ be total final use. Then let the final-use share of industry $j$ be denoted by

\begin{equation}
	\gamma_j = \frac{f_j}{FINAL}.
\end{equation}

Last, create a $J+2$ by one vector $\Gamma$ which has $\gamma_j$ in row $j$ for the first $J$ rows, and zeroes in the last two rows. Those last two rows represent the final-use shares of the capital and labor industries, which are zero as those two factors are used solely as inputs by other industries. The structure of $\Gamma$ is

\begin{equation}
	\Gamma' = 
	\begin{bmatrix}
		\gamma_1 &
		\gamma_2 &
		\cdots &
		\gamma_J &
		0 &
		0
	\end{bmatrix} \label{EQ_Gamma}
\end{equation}

Given this information, one can calculate the vector of what BF call ``cost-based Domar weights'', $E$, for each industry. 

\begin{equation}
	E = \Gamma' (I - \Lambda)^{-1} \label{EQ_E}
\end{equation}

where $I$ is a $J+2$ square identity matrix, and $(I - \Lambda)^{-1}$ is the Leontief inverse matrix of the expanded input-output matrix. 

The structure of $E$ is as follows,

\begin{equation}
	E = 
	\begin{bmatrix}
		\epsilon_1 &
		\epsilon_2 &
		\cdots &
		\epsilon_J &
		\epsilon_K &
		\epsilon_L
	\end{bmatrix}
\end{equation}

where $\epsilon_1 ... \epsilon_J$ are the cost-based Domar weights for the regular industries, and $\epsilon_K$ and $\epsilon_L$ are the cost-based Domar weights for capital and labor. What BF prove is that in this setting $\epsilon_K$ and $\epsilon_L$ are the elasticity of aggregate output with respect to capital and labor, respectively. As this calculation holds for any period $t$, the elasticities are more properly denoted $\epsilon_{Kt}$ and $\epsilon_{Lt}$.

It is worth considering the intuition behind this result. In equation (\ref{EQ_E}) the term $(I - \Lambda)^{-1}$ is the Leontief inverse. Let $\ell_{ij}$ denote the typical element of the Leontief inverse. $\ell_{ij}$ captures the elasticity of output in industry $i$ with respect to a productivity shock in industry $j$, accounting for all the input-output linkages joining them.\footnote{\cite{cts2018} provide a very nice introduction to production networks and the interpretation of the Leontief inverse.}

In the case of capital and labor, these ``industries'' have no final use and only serve as suppliers of an input to other industries. A productivity shock to these factor ``industries'' is nothing more than an increase in their supply. Hence the values of $\ell_{iK}$ and $\ell_{iL}$ show us the elasticity of output in industry $i$ with respect to the supply of capital or labor, respectively. This elasticity incorporates not just the direct effect of more capital or labor on output in industry $i$ (which is captured by the cost shares $\lambda_{iK}$ and $\lambda_{iL}$), but also incorporates the indirect effect of increased factor inputs on the output of suppliers to industry $i$, on the output of suppliers to the suppliers of industry $i$, and so on. 

BF show that that $\ell_{iK}$ and $\ell_{iL}$ measure the elasticity of real output (as opposed to revenue) with respect to capital and labor so long as the original entries in $\Lambda$ are \textit{cost} shares and not value-added shares. This is the same insight from \cite{hall1988,hall1990}, \cite{basufernald}, and \cite{fernaldneiman2011} regarding the use of cost shares to measure elasticities, but applied in a dis-aggregated manner to an economy with input-output linkages between industries. The BF structure is open, in the sense that some amount of intermediate good spending may be imported, rather than produced domestically (and some output of domestic industries may be exported). Imports and exports influence the final use shares, $\gamma_j$, of industries, and thus influence the calculation of the elasticities. The elasticities $\epsilon_{Kt}$ and $\epsilon_{Lt}$ are the elasticity of domestic output with respect to those factors of production.\footnote{In the Appendix I consider an alternative approach that excludes imported intermediates from the calculations. This produces estimates of the elasticities that are very similar to the baseline including imported intermediates.}

The aggregate effect of an increase in capital or labor depends on the individual terms $\ell_{iK}$ and $\ell_{iL}$ weighted by the final-use share of each industry. From the matrix multiplication in equation (\ref{EQ_E}) the aggregate elasticity with respect to each input can be expressed as

\begin{eqnarray}
	\epsilon_{Kt} &=& \sum_{i \in J} \gamma_{it} \ell_{iKt} \\ \nonumber
	\epsilon_{Lt} &=& \sum_{i \in J} \gamma_{it} \ell_{iLt}. \label{EQ_epsilonK}
\end{eqnarray}

This breakdown of the elasticities will be useful later in decomposing the change in the elasticities over time. Overall, the BF structure makes it possible to calculate the elasticity of aggregate output with respect to aggregate capital or labor without an aggregate production function, with rich input/output structures, and with arbitrary markups at the industry level.

\section{National accounts data}\label{SEC_accounts}
The calculation of $\epsilon_{Kt}$ and $\epsilon_{Lt}$ from equation (\ref{EQ_E}) is straightforward in theory, but not in practice. The well-known issue is the disconnect between what is reported in the national accounts (e.g. gross operating surplus) and what is necessary for the calculation (e.g. the cost of capital). This disconnect is why in the following section I will build bounds for estimates of $\epsilon_{Kt}$ (and $\epsilon_{Lt}$) based on different ways of reconciling the national accounts with the needs of the calculation. 

Prior to those main results I explain the three main U.S. data sources, set notation regarding the national accounts, and explain the process I used to merge together various sources to create industry-level data that spans 1948-2018 for the United States.

The first source of data are input/output tables. These are the Bureau of Economic Analysis (BEA) Use and Make Tables, before redefinitions, at producer value \citep{beaio}. These annual Tables provide information on the costs of intermediate commodity $m$ purchased by industry $i$ in year $t$ (Use Table), and the amount of commodity $m$ produced by industry $i$ in year $t$ (Make Table). I use a standard procedure to combine the information in the Use and Make Tables to arrive at the cost of inputs from industry $j$ purchased by industry $i$ in year $t$, $COST_{ijt}$. Those cost terms are used as in equation (\ref{EQ_lambdaij}) to find the cost shares $\lambda_{ijt}$ that make up the matrix $\Lambda$ in equation (\ref{EQ_Lambda}).\footnote{The Appendix contains a full description of the procedure used to combine the Use and Make Tables to form consistent industry-by-industry measures of intermediate costs. In addition, the Appendix explains how my use of the ``before redefinition'' Tables generates very small numerical differences in elasticity estimates compared to estimates made using the ``after redefinition'' Tables provided by the BEA 1997-2018, while allowing me to extend estimates from 1948-2018.} The Tables also provide information on the final use of each industry, $f_i$ according to the notation developed above, and which goes into the formation of the vector $\Gamma$ in equation (\ref{EQ_Gamma}). Finally, the Tables provide information on the value-added of each industry $i$, which I denote here as $VALU_{it}^{IO}$, where the $IO$ superscript refers to the source of this data.

Industries in the Use Tables are all classified according to the NAICS 2012 system, but due to data limitations the BEA provides the tables at different levels of aggregation depending on the year. For 1948-62 they report 46 industries, for 1963-1996 65 industries, and for 1997-2018 71 industries. 

Given the input/output information from the BEA, the second source of data are industry-level components of value-added from the BEA national income and product accounts \citep{beasection6,beahistind}. Specifically, I collect measures of value-added, $VALU_{jt}^{NIPA}$, labor compensation, $COMP_{jt}^{NIPA}$, proprietors income, $PROP_{jt}^{NIPA}$, and taxes and subsidies, $TAX_{jt}^{NIPA}$ for each industry $j$ in year $t$. The superscript $NIPA$ refers to the source of this data. Last, note that this data is sub-scripted by $j$ (not $i$) to indicate that the industrial classification of this NIPA data may be different than the industrial classification in the IO table.

To impute labor compensation data (for example) from NIPA to the IO table, I use the following equation,

\begin{equation}
	COMP_{it}^{IO} = VALU_{it}^{IO} \times \frac{COMP_{jt}^{NIPA}}{VALU_{jt}^{NIPA}}. \label{EQ_match}
\end{equation}

To implement this I need to match each industry $i$ in the IO table to an appropriate industry $j$ in the NIPA data. Given that match, I use the ratio of compensation to value-added in the NIPA industry to impute the size of compensation in the IO industry. A similar expression is used for both proprietors income and taxes and subsidies. 

Table \ref{TAB_series} documents the classification schemes used by different sources (IO, NIPA) for different years. For 1997-2018 the NIPA data \citep{beasection6} are classified according to the NAICS 2012 system, and thus it is possible to match an industry $i$ in the IO table to the exact same industry $j$ in the NIPA data. Further, the value-added reported in the two sources are identical for these years, and hence the expression above devolves to $COMP_{it}^{IO} = COMP_{jt}^{NIPA}$ (and similar for the other components of value-added). 

For 1948-1996, however, NIPA data are not reported according to the NAICS 2012 classification. Industry-level data from the \textit{Historical Industry Accounts Data} \citep{beahistind} is classified by either SIC 1972 or SIC 1987 industries, depending on the year. I continue to use the equation above to find compensation (and proprietors income and taxes) in industry $i$ of the IO table, but this now requires making assumptions about which industry $j$ in the NIPA data provides an appropriate match to industry $i$ in the IO data in year $t$.

I rely on crosswalks between SIC 1972, SIC 1987, and NAICS 2012 classifications, and my own judgment to make these matches. A straightforward example from 1960 would be using ``Transportation by air'' (SIC 1972 code 45) as industry $j$ from NIPA to match to ``Air transportation'' (NAICS 2012 code 481) as industry $i$ in the IO table.

The matching for a given year is not always one-for-one, and there are NIPA industries $j$ (e.g. SIC 1972 code 73 ``Business services'') whose ratios are used for multiple industries $i$ in the I/O table (e.g. NAICS code 561 ``Administration and support services'', NAICS code 55 ``Management of companies'', etc.). There are also situations where I have aggregated the data from NIPA industries (e.g. SIC 1972 codes 63 ``Insurance carriers'' and 64 ``Insurance agents, brokers, and service'') and then matched this aggregate to an industry $i$ in the IO table (e.g. NAICS code 524 ``Insurance carriers and related activities''). 

Full details of the matching are available in the Appendix. I have experimented with a variety of different reasonable choices for matching, and have not found any that change the results of the paper in an appreciable way.\footnote{Code and instructions are available on my website to the reader who wishes to experiment with different matching assumptions between the NIPA and IO table sources.}

The final data sources I use are the Fixed Asset Accounts Tables of the BEA \citep{beacap,beagov}. These provide information on the size of the capital stock of type $k$ in industry $j$ at years $t$, $K_{jkt}$, the amount of depreciation, $DEPR_{jkt}$, and investment spending, $INV_{jkt}$. The three types of capital $k$ reported are structures, equipment, and intellectual property. This fixed asset data is reported according to the NAICS 2012 classification, and so can be matched directly to the IO table industries.\footnote{There are minor discrepancies between the NAICS classifications in the Fixed Asset Accounts and the IO tables. These are straightforward to manage in that the fundamental classification system is the same. Details are in the Appendix.} 

Once combined the dataset contains yearly information from 1948-2018, at the NAICS 2012 classification level, of industry-level information on intermediate costs, value-added, labor compensation, proprietors income, taxes and subsidies, depreciation of capital (by type), investment in capital (by type), and the stock of capital (by type).

\section{Bounding the aggregate elasticities}\label{SEC_bounding}
As mentioned above, there is a disconnect in the presentation of data in the national accounts and the requirements of the calculation of $\epsilon_{Kt}$ and $\epsilon_{Lt}$ from equation (\ref{EQ_E}). In short, this is an issue of how to split gross operating surplus into labor costs (part of proprietors income), capital costs, and economic profits. As there is no correct answer for how to do this, my approach is to construct several estimates of $\epsilon_{Kt}$ and $\epsilon_{Lt}$ based on different assumptions.

Two of these assumptions will form what I consider to be natural bounds on the size of $\epsilon_{Kt}$ and $\epsilon_{Lt}$. Assuming that there are zero economic profits in the economy will maximize the implied cost of capital imputed from the national accounts data, and hence lead to an upper bound for $\epsilon_{Kt}$ (and a lower bound for $\epsilon_{Lt}$). On the other hand, we know that industries experienced the depreciation of existing capital and their capital costs are at least this large. Using depreciation costs to impute the cost of capital will therefore give a lower bound for $\epsilon_{Kt}$ (and an upper bound for $\epsilon_{Lt}$). 

Neither of these bounds are necessarily unbreakable. Depreciation is an estimate made by the BEA, and hence may not be an accurate measure of those costs. If the financing costs for capital are negative, this would also imply total capital costs could be below depreciation. Alternatively, in the presence of \textit{negative} economic profits the implied cost of capital could be even higher than supposed with the zero-profit assumption. Nevertheless, the other assumptions regarding capital costs, such as a user cost calculation, do tend to fall within the bounds I establish.

\subsection{Labor costs, proprietors income, and taxes}
Prior to detailing how capital costs are handled to form bounds on the elasticities, I describe how I allocate proprietors income to different factors of production. I follow \cite{gommerupert2004} and assign a share of proprietors income and production taxes to labor equal to the ratio of reported compensation to non-proprietors value-added. Hence the cost of labor is calculated according to:

\begin{equation}
	COST_{iLt} = COMP_{it} + (PROP_{it} + TAX_{it})\left(\frac{COMP_{it}}{VALU_{it}-TAX_{it}-PROP_{it}}\right), \label{EQ_prop}
\end{equation}

where $COMP_{it}$ is reported labor compensation in industry $i$ at time $t$, $PROP_{it}$ is proprietors income, $VALU_{it}$ is value-added, and $TAX_{it}$ is taxes and subsidies. \cite{gommerupert2004} argue that this provides a more accurate representation of the labor component of proprietors income than using the number of self-employed workers and a measure of average wages, as proprietors are likely to be high productivity (and hence high wage). This also apportions production taxes to labor in the same manner.

In the Appendix I show variations on this assumption where I either assign all proprietors income as labor costs (i.e. $COST_{iLt} = COMP_{it} + PROP_{it}$) or all proprietors income as capital cost (i.e. $COST_{iLt} = COMP_{it}$). In the former case the results are quite similar to the baseline using the Gomme and Rupert approximation. In the latter case the implied costs of capital are higher, implying higher measures of $\epsilon_{Kt}$ (and lower estimates of $\epsilon_{Lt}$). However, this seems an unlikely case, given the widespread use of adjustments to proprietors income that assign much of it as labor income. Recent work by \cite{syzz2019} also suggests that much of what may be reported as gross operating surplus by pass-through firms is ultimately a payment to labor, further reinforcing that treating most proprietors income as labor income as in Gomme and Rupert is the most reasonable choice.

\subsection{Boundaries for capital costs}
With the labor cost determined it remains to impute capital costs, $COST_{iKt}$, for each industry in each year, and to calculate the values of the various elasticities. In what follows, I only discuss values for $\epsilon_{Kt}$, to be concise. In each case I also obtain estimates for the labor elasticity, $\epsilon_{Lt}$. In every case one can use the values of $\epsilon_{Kt}$ to infer values for the labor elasticity of $\epsilon_{Lt} = 1 - \epsilon_{Kt}$. The annual estimates for all the elasticities, under each of the assumptions about capital costs, are available in the Appendix.

\subsubsection{No profit assumption}
The first assumption is that there are zero economic profits. This means that gross operating surplus minus any adjustments for proprietors income and taxes represents a payment to capital. To be specific, for this assumption I set

\begin{equation}
	COST^{NoProf}_{iKt} = VALU_{it} - COST_{iLt},  \label{EQ_noprofit}
\end{equation}
where $COST_{iLt}$ are labor costs as explained in (\ref{EQ_prop}). Given this cost of capital for an industry, I can then calculate the capital cost share, $\lambda_{iKt}$. In addition, I can calculate the cost shares of all other intermediates, $\lambda_{ijt}$, and the cost shares of labor, $\lambda_{iLt}$.

With the cost shares in hand, it is straightforward to calculate $\epsilon_{Kt}^{NoProf}$ given the formula in (\ref{EQ_E}). As the aggregate capital elasticity is effectively a weighted average of the industry-specific capital cost shares, by using the no-profit assumption $\epsilon_{Kt}^{NoProf}$ I have an upper bound on the aggregate capital elasticity.

Figure \ref{FIG_cap_comparison} plots the estimated values of $\epsilon_{Kt}^{NoProf}$ over time for the United States as the heavy black dashed line. In Figure \ref{FIG_cap_comparison} one can see that the no-profit capital elasticity begins at \basefirstnoprofit \ in 1948 and rises with mild fluctuations to a value of \baselastnoprofit \ by 2018. This no-profit upper bound for the capital elasticity tracks the value of 1/3 (with a mild dip in the 1970s) from 1948-1995. The value of 1/3 would only be appropriate in those years if one believed there were zero economic profits in the economy. After 1995, there is shift up in the no-profit upper bound to around \baselatenoprofit, on average, making the value of 1/3 appear more plausible. Table \ref{TAB_scenario} provides summary statistics of the estimated capital elasticities

\subsubsection{Depreciation costs only}
The second estimates of the aggregate elasticities are made using deprecation to impute the cost of capital. Depreciation by itself misses costs associated with the on-going financing of the capital stock by firms, but has the advantage of being reported on an industry-by-industry basis. We know that industries face \textit{at least} a depreciation cost for their capital. From that perspective, using depreciation provides a plausible lower bound for capital costs. The drawback of this approach is that depreciation is imputed by the BEA for each industry, based on historical investment spending and assumed depreciation schedules for various types of capital.\footnote{This also presumes that the required real return on capital is positive, as if that return were negative than the overall cost of capital could be less than depreciation. In the user cost assumption discussed below I allow for this possibility.} Hence a depreciation-based estimate of $\epsilon_{Kt}$ is itself subject to some uncertainty, and this lower bound is not as ``firm'' as the upper bound derived from the no-profit assumption. 

In terms of the structure outlined above, I measure capital costs in industry $i$ as follows

\begin{equation}
	COST^{Depr}_{iKt} = DEPR_{it}. 
\end{equation}

As this point the logic is identical to the prior sub-section. These costs allow me to calculate cost shares for each industry, and those cost shares are used in equation (\ref{EQ_E}) to calculate the aggregate capital elasticity, $\epsilon_{Kt}^{Depr}$.

In Figure (\ref{FIG_cap_comparison}) the capital elasticity $\epsilon_{Kt}^{Depr}$ is plotted from 1948-2018 as the solid black line. As expected, this series lies everywhere below the no-profit estimates. The estimated elasticity begins at \basefirstdepr \ in 1948 and finishes at \baselastdepr \  in 2018, for an increase of 0.09 that is very similar to the increase in the no-profit upper bound. 

From the figure it is also apparent that the gap between the no-profit upper bound and the depreciation-based lower bound remains roughly constant. In the Figure that range is shaded in light gray to indicate the plausible values for $\epsilon_{Kt}$ in any given year. Note that this range is not a confidence interval and nothing about it implies that the actual capital elasticity lies in the middle of the range. In particular, as depreciation costs are imputed, it is plausible that the lower bound for $\epsilon_{Kt}$ is in fact higher or lower than what is plotted in the Figure.

\subsection{Alternative capital cost estimates}
The prior sub-section showed the boundaries over time, and here I present two more estimates of the capital elasticity that do not make sense as bounds, but provide information on the actual path of the elasticity and whether those bounds are sensible.

\subsubsection{Investment costs}
Here I use observed investment spending by each industry as the measure of capital costs. One way this imputation may be sensible is to consider an economy following a ``Golden Rule'', where all capital income is used to purchase capital goods. As capital income to one person represents a capital cost to another, investment spending would measure capital costs.

An additional advantage of the investment cost assumption is that this data is measured directly from industry level spending. Depreciation costs by themselves are estimated by the BEA based on depreciation schedules that may not accurately reflect true industry experience. Arguably the use of observed investment spending to measure capital costs may be the choice with the fewest assumptions built in.

In this case the cost of capital is measured as follows,

\begin{equation}
	COST^{Inv}_{iKt} = INV_{it}. 
\end{equation}

Once again, the logic at this point is standard. Using these costs I obtain cost shares for capital, labor, and intermediates, and using (\ref{EQ_E}) I can calculate $\epsilon_{Kt}^{Inv}$. 

Figure (\ref{FIG_all_comparison}) plots the estimated capital elasticity from 1948-2018 with the line marked with x's, as well as the original bounds. This estimate begins at \basefirstinv \ in 1948 and runs to \baselastinv \ in 2018, demonstrating a less dramatic increase that either the depreciation or no-profit bounds. Based on investment costs, the capital elasticity ends up at the lower depreciation-based bound around the time of the Great Recession in 2009 and remains close to that lower bound until 2018. This is consistent with investment spending by industry that acts solely to replace depreciating capital in that period.

Of note, the investment cost elasticity estimate remains everywhere inside the bounds set by the deprecation-only and no-profit estimates. The relatively small increase over time in the investment cost elasticity estimates reminds one that the true capital elasticity may well move between the bounds over time and does not necessarily increase just because the bounds do. In addition, the capital elasticity based on investment costs is everywhere below the value of 1/3.

\subsubsection{User cost of capital}
As a last alternative, I turn to the standard user cost of capital calculation of \cite{halljorg1967}. This has been used extensively to estimate the cost of capital, including in recent work on labor and capital's share of aggregate GDP \citep{Barkai000}, with the downside of needing to make several assumptions about the financing costs facing industries and expectations of inflation of capital goods. 

The calculation of the overall cost of capital for an industry $i$ is more complex than the prior assumptions. First, I allow for three types of capital goods - structures, equipment, and intellectual property - which is available from the BEA capital stock data \citep{beacap,beagov}. Each industry $i$ has a stock of capital of each type $j$ at time $t$, $K_{ijt}$. Each industry $i$ also faces a rental rate for capital of type $j$ at time $t$, $R_{ijt}$, and these rental rates are allowed to vary by industry.

Overall, the cost of capital to industry $i$ at time $t$ is

\begin{equation}
	COST^{User}_{iKt} = \sum_{j \in st,eq,ip} K_{ijt} R_{ijt}.
\end{equation}

The rental rate for each type of capital in a given industry is given by

\begin{equation}
	R_{ijt} = (Int_{it} - E[\pi_{ijt}] + \delta_{ijt})\frac{1-z_{jt} \tau_t}{1-\tau_t}
\end{equation}

where $Int_{it}$ is the nominal interest cost of financing facing industry $i$ at time $t$, explained below. $E[\pi_{ijt}]$ is the expected inflation in the price of capital type $j$ for industry $i$ at time $t$, and $\delta_{ijt}$ is the cost of depreciation. The term $z_{jt}$ is the depreciation allowance for taxation of capital type $j$ at time $t$, and $\tau_t$ is the effective corporate tax rate at time $t$.

In this baseline, the expected inflation rate $E[\pi_{ijt}]$ for a given capital type $j$ in industry $i$ at time $t$ is just observed inflation rate in $t$, based on the price indices by capital type in the BEA fixed asset accounts \citep{beacap,beagov}. There are only small changes to the results if I instead proxy expected inflation using forward-looking or backward-looking inflation over different spans (1-year, 3-year, 5-year).

The nominal interest rate facing industry $i$ at time $t$ is calculated as a weighted average of market interest rates for different types of financing (e.g. corporate bonds, equity, mortgages), denoted by $Int_{mt}$, and the weights vary by industry. Formally,

\begin{equation}
    Int_{it} = \sum_m s_{imt} Int_{mt}
\end{equation} 

where $s_{imt}$ is the share of financing of type $m$ in industry $j$, and $Int_{mt}$ is the observed interest rate on that type of financing. 

While this is industry-specific, it primarily differentiates between government, housing, and the private sector. Full details are available in the Appendix on sources for the shares and rates used. A brief summary is that private industries are financed using a combination of corporate AAA bonds, corporate Baa bonds, short-term loans, and equity, with the shares ($s_{imt}$) determined from industry-level balance sheets provided by the integrated macroeconomic accounts \citep{beaimap}. Housing is assumed to be financed using 30-year mortgages. Government ``industries'' are assumed to be financed using 10-year Treasury bonds (federal) or municipal bonds (state/local).\footnote{Treating government in this manner assumes that it acts similar to private industries in making decisions on capital use. An alternative is to assert that user costs of capital to government industries are equal to the reported depreciation, which would be consistent with how the BEA calculates value-added for government. Doing so does not alter the results in any appreciable way.}

Given the values of $Int_{it}$ I am able to calculate the rental rate of capital facing each industry $i$ for each capital type $j$, $R_{ijt}$, and then the overall cost of capital for industry $i$ at time $t$, $COST^{User}_{iKt}$. With those costs of capital, I can then proceed in the same manner as before, and calculate the elasticity $\epsilon_K^{User}$ using equation (\ref{EQ_E}).

In Figure (\ref{FIG_all_comparison}) the series of $\epsilon_K^{User}$ is plotted from 1948-2018 marked by o's. This is far more variable over time than the bounds set by depreciation and no-profit assumptions, as well as more variable than the estimates based on investment costs. However, the user cost of capital estimates stay for the most part inside the bounds set by depreciation and no-profit.

There are notable exceptions. In 8 years (1950, 1973, 1974, 1977, 1978, 2004, 2005, and 2013) the user cost of capital is below the bound set by the depreciation cost estimates. The observations in the 1970s are due to very high inflation of all capital goods, which in the user cost calculation results in very low rental rates and hence a low cost to capital. The 2004 and 2005 observations are due to very high inflation in structures. 1950 and 2013 appear to be a combination of slightly higher inflation in structures and low financing rates. Nevertheless, these deviations below the bound of depreciation cost estimates are not large and appear to be short-lived. 

On the other end, there is a continuous stretch from 1981 through 1992 where the user cost estimates are above the upper bound set by the no-profit estimates. These are due to the relatively high nominal rates on financing during this period and the lower values observed for inflation on all capital goods. These deviations are not large after 1984. If the user cost estimates of the elasticities are correct, then they would imply negative economic profits during this period. Alternatively, the user cost calculations may not be accurately representing the cost of capital in this period.

Regardless, in 51 of the 71 years reported the estimate of $\epsilon_K^{User}$ falls between the bounds denoted by the no-profit and depreciation costs estimates. Over time, the trend of the user cost estimates appears to track the trend of the investment cost estimates, and both imply an elasticity that is below 1/3 for much of this time period. 

\section{Trends and relationships to aggregate values}\label{SEC_trends}
None of the four estimated series of $\epsilon_{Kt}$ in Figure \ref{FIG_all_comparison} are ``right''. Without direct measurement of the cost of capital faced by industries, and better information on the split of proprietors income, any estimate of the aggregate elasticities is necessarily based on some assumptions. However, the combination of the various estimates can be used as guidance regarding plausible values of $\epsilon_{Kt}$ (and $\epsilon_{Lt}$) over time. In addition, these estimates can be compared to ``naive'' estimates based on economy-wide ratios to see how informative those ratios are.

\subsection{Summary and trends}
Table \ref{TAB_scenario} summarizes the values of $\epsilon_{Kt}$ in Panel A (``Baseline'') for the four different assumptions regarding capital costs from the prior section. The mean, median, minimum, and maximum value from 1948-2018 of the capital elasticity are displayed in the first four columns. As can be seen, under the no-profit assumption the mean and median values of $\epsilon_{Kt}$ are highest, and it has the highest minimum value as well. In comparison, the depreciation assumption has the lowest mean and median, and the lowest maximum value of all the assumptions used. 

Based on the median values, this implies that $\epsilon_{Kt}$ lies in a range of \basemeddepr-\basemednoprofit, and that the typical assumption of $\epsilon_{Kt}$ being equal to one-third is likely too big unless one is focused only on the period from around 2000-2018.

There is a suggestion in Figure \ref{FIG_all_comparison} that the range of plausible capital elasticities may have risen over time. The final three columns of Table \ref{TAB_scenario} provide more tangible estimates of that rise. For each assumption on capital costs, I ran the following simple OLS regression

\begin{equation}
	\epsilon_{Kt} = \beta_0 + \beta_1 t + \nu,
\end{equation}

where $t$ is the year and $\nu$ is a noise term. The estimate $\hat{\beta}_1$ was used to construct the fitted increase in $\epsilon_{Kt}$ over time for that assumption, $\Delta \hat{\epsilon}_{48-18} = 70 \times \hat{\beta}_1$. This fitted change is reported in column (5), the value of $\hat{\beta}_1$ in column (6), and the R-squared of the regression in column (7). 

One can see that for all four assumptions regarding capital costs in Panel A of Table \ref{TAB_scenario}, the fitted change was positive. For the no-profit assumption, the fitted value of the capital elasticity rose by 0.069, while for the depreciation cost assumption it rose by 0.10. The investment cost assumption had the smallest trend increase, or only 0.053. 

None of this constitutes proof that the aggregate capital elasticity rose over time. It is quite plausible that the capital elasticity stayed constant over time but the noisy nature of the data has created an impression of trends. But if we accept the increase in the bounds as evidence of an upward trend in $\epsilon_{Kt}$, note that this elasticity is almost certainly below 1/3 from 1948-1995, and below 0.4 from 1995-2018. It is also plausible that the capital elasticity was around 1/4 the entire time, given the evidence from the investment cost assumption.

\subsection{Comparison to aggregate ratios}
The estimates of the aggregate elasticity are built using industry-level data with input-output relationships, but it is informative to compare that elasticity to aggregated data on capital and labor costs. Such aggregate data is what has been typically used to estimate these elasticities in the past, and has the advantage of being much more readily available.

Using the notation developed above, define aggregate capital costs as a share of all factor costs (labor plus capital), $s^{Cost}_{Kt}$, as follows,

\begin{equation}
	s^{Cost}_{Kt} = \frac{\sum_{j=1}^{J} COST_{jKt}}{\sum_{j=1}^{J} COST_{jKt} + \sum_{j=1}^{J} COST_{jLt}}. \label{EQ_scost}
\end{equation}

A second aggregate ratio to consider is aggregate capital costs as a share of value-added, $s^{VA}_{Kt}$, defined as

\begin{equation}
	s^{VA}_{Kt} = \frac{\sum_{j=1}^{J} COST_{jKt}}{\sum_{j=1}^{J} va_{jt}}, \label{EQ_sva}
\end{equation}
where $va_{jt}$ is the value-added of industry $j$ at time $t$.

Note that both of these aggregate ratios still depend on the choice of assumption regarding capital costs. Table \ref{TAB_ratios} provides summary statistics for both $s^{Cost}_{Kt}$ and $s^{VA}_{Kt}$ in Panel A. Under the no-profit assumption in the first row, columns (1)-(4) the mean value of $s^{Cost}_{Kt}$ is \basemeannoprofit \ and the median \basemednoprofit, and ranges from a minimum of \baseminnoprofit \ to a maximum of \basemaxnoprofit \ between 1948 and 2018. In columns (5)-(8) the summary statistics for $s^{VA}_{Kt}$ are identical for the no-profit assumption, by construction. With zero economic profits the denominators of the two ratios are identical. Further, if one compares the ratios $s^{Cost}_{Kt}$ and $s^{VA}_{Kt}$ under the no-profit scenario to the estimates of $\epsilon_{Kt}$ under the same no-profit assumption in Panel A of Table \ref{TAB_scenario}, one can see that they are identical. As \cite{bfshortnote,bfprodge} derive, with zero economic profits the elasticity is equal to capital's share of value-added. 

The consistency across $s^{Cost}_{Kt}$, $s^{VA}_{Kt}$, and $\epsilon_{Kt}$ does not hold under the other assumptions regarding capital costs, as they all imply some amount of economic profits are being earned. One can examine the ratios in Panel A of Table \ref{TAB_ratios}. Reading down the rows, there are distinctions between $s^{Cost}_{Kt}$ and $s^{VA}_{Kt}$ in the different scenarios, and the summary statistics of these ratios do not match the summary statistics of $\epsilon_{Kt}$ from Table \ref{TAB_scenario}. Economic profits explain why $s^{VA}_{Kt}$ is different than the cost ratio $s^{Cost}_{Kt}$ for these scenarios, and I discuss this further in the next sub-section.

All that said, the aggregate cost ratios provide a crude approximation of the capital elasticity under different capital cost assumptions. Figure \ref{FIG_ratio_comparison} plots the estimated values of $\epsilon_{Kt}$ for all years under three assumptions regarding capital costs against the corresponding ratio $s^{Cost}_{Kt}$ for that assumption in the same year. The no-profit assumption is excluded because the elasticity and cost ratio are identical by construction.

What can be seen is that the capital elasticity estimates lie generally above the corresponding ratios, with the 45-degree line denoted by dashes.\footnote{The labor elasticities all lie generally below their corresponding cost ratio.} The average difference between elasticity and ratio ($\epsilon_{Kt} - s^{Cost}_{Kt}$) under the depreciation cost assumption is 0.030, and the root squared difference under the same assumption is 0.032.\footnote{By ``root squared difference'' I mean $\sqrt{\sum_t (\epsilon_{Kt} - s^{Cost}_{Kt})}$, calculated over the 70 annual observations.}. For the investment cost assumption, the average difference is 0.020 and the root squared difference is 0.024. Finally, for the user cost assumption the average difference is 0.007 and the root squared difference is 0.019. As can be seen, the differences tend to be largest under the depreciation cost scenario. Overall, the size of the differences suggests that one could derive a reasonable approximation from the ratios alone, which may be useful in the absence of an input/output table.

\subsection{Elasticities and Markups}
The differences between $\epsilon_{Kt}$ and $s^{Cost}_{Kt}$ in the prior sub-section are due to economic profits. I provide a more thorough theoretical comparison of $\epsilon_{Kt}$ and $s^{Cost}_{Kt}$ in the Appendix, but the general intuition is straightforward. $s^{Cost}_{Kt}$ depends on industry-specific capital cost shares and the allocation of total costs across industries. Markups skew the allocation of total costs across industries, and hence skew the value of $s^{Cost}_{Kt}$. When markups are positively correlated with industry-specific capital cost shares, there are fewer costs expended on the high-markup/capital-intense industries, and hence $s^{Cost}_{Kt}$ is pushed below $\epsilon_K$. Furthermore, when industries that have high capital shares of costs tend to be upstream in the input/output structure, multiple marginalization lowers their share of total costs, also pushing $s^{Cost}_{Kt}$ below $\epsilon_K$. In general, the larger are markups overall the larger are these effects, and hence the gaps in Figure \ref{FIG_ratio_comparison} between $\epsilon_{Kt}$ and $s^{Cost}_{Kt}$ are generally largest for the depreciation cost scenario.

Given that markups explain the gap between $\epsilon_{Kt}$ and $s^{Cost}_{Kt}$, are the markups implied in my scenarios reasonable? In each scenario regarding capital costs it is possible to calculate an aggregate gross output markup as
\begin{equation}
    \mu^{GO}_{t} = \frac{\sum_{j=1}^{J} GO_{jt}}{\sum_{j=1}^J COST_{jMt} + COST_{jKt} + COST_{jLt}}. \label{EQ_markup}
\end{equation}
In practice this follows \cite{edmondetal2018} and computes a cost-weighted average of industry-specific gross output markups.

Figure \ref{FIG_markup} plots the estimates of $\mu^{GO}_t$ associated with various assumptions regarding capital costs. The no-profit assumption, by construction, implies a markup of $\mu^{GO}_t = 1$ in each year, and is shown as the horizontal dashed line in the Figure. In contrast, the depreciation lower bound on capital costs implies that there are positive economic profits, and it forms an upper bound on the markup in each year.

The implied gross output markups here range from 1 to about 1.15 over time. The series using investment costs to measure capital costs suggests a markup of around 1.05 until around 1980, and then an increase to just over 1.10 by 2010. These numbers are consistent with the evidence reviewed in \cite{basu2019} on aggregate gross output markups.\footnote{If I alternatively calculate value-added markups, the lower bound is again 1, of course, and the depreciation upper bound on markups is about 1.2-1.3 throughout the period under study. There is a similar increase in the value-added markup implied by the investment cost estimates from around 1.1 in 1980 to about 1.25 in 2010. These estimates are also consistent with evidence cited in \cite{basu2019}.}

These bounds imply that economic profits as a share of gross output were between zero and about 10-12\% from 1948-2018. Putting that in terms of economic profits as a share of value-added (which depends on how large spending on intermediate inputs is compared to gross output) results in a range of zero (no profit scenario) to about 18-22\% (depreciation cost scenario) in the same time span. This economic profit share of value-added range is spans the results in \cite{Barkai000}, who finds the share close to zero around 1985 and rising to around 15\% by 2015. 

Similar to the elasticities, the point here is not to provide a single estimate of the gross output markup or profit share of value-added, but instead to show their bounds, and that they are consistent with other external estimates. The range of markups suggested by my estimates appears reasonable. Those markups explain the gaps between $\epsilon_{Kt}$ and $s^{Cost}_{Kt}$ in the prior subsection and suggests that those gaps are reasonable as well.

\subsection{Accounting for change}
The bounds on $\epsilon_{Kt}$ rose over time. By themselves, these changes in the bounds do not necessarily imply that the actual values of $\epsilon_{Kt}$ (or $\epsilon_{Lt}$) changed. But it seems worth exploring what drove the changes in the bounds, as they imply shifts in the aggregate elasticities were plausible.

To account for the change in bounds, return to equation (\ref{EQ_epsilonK}) and the expression showing how $\epsilon_{Kt}$ is a weighted sum of entries from the Leontief inverse, with weights given by shares of final use,

\begin{equation}
	\epsilon_{Kt} = \sum_{j=1}^J \gamma_{it} \ell_{iKt}. \nonumber
\end{equation}

To track the changes in $\epsilon_{Kt}$ over time I perform an \cite{op1996} decomposition on this summation, yielding

\begin{equation}
	\epsilon_{Kt} = \overline{\ell}_{Kt} + \sum_{j=1}^J (\gamma_{it} - \overline{\gamma}_{t})(\ell_{iKt}-\overline{\ell}_{Kt}), \label{EQ_op}
\end{equation}

where $\overline{\ell}_{Kt}$ is the unweighted mean of the Leontief elements for capital. This mean industry-level elasticity shows how sensitive industries are to capital, ignoring their share of final use. Tracking this over time will indicate whether industries in general were becoming more or less sensitive to the use of capital.

The summation term above is the ``covariance'' of final-use shares and Leontief elements. When positive, it indicates that industries that are more sensitive to capital (e.g. have $\ell_{iKt}$ above average) also tend to be large in final-use terms. When negative, it indicates that industries sensitive to capital are relatively small. Tracking this covariance term over time will show whether capital-sensitive industries were becoming larger or smaller.

Figure \ref{FIG_op_comparison} plots the values of $\epsilon_{Kt}$ and $\overline{\ell}_{Kt}$ for both the upper and lower bounds of the elasticity, which are determined by the no-profit and depreciation cost assumptions on capital costs. The covariance is not plotted separately but can be inferred from examining the Figure, as it is the gap between plotted series. 

For the no-profit upper bound, it is apparent that the mean industry-level elasticity, $\overline{\ell}_{Kt}$, drove the drift up over time. The gap between $\epsilon_{Kt}$ and $\overline{\ell}_{Kt}$ is accounted for by the covariance term in (\ref{EQ_op}), which is positive but small in absolute size throughout. The upper bound on the aggregate elasticity rose over time because, on average, most industries were getting more sensitive to capital. 

This story is repeated with the depreciation cost lower bound. Again the mean industry-level elasticity, $\overline{\ell}_{Kt}$, lies everywhere below the aggregate elasticity, $\epsilon_{Kt}$, which implies again a small positive covariance term. The drift upward is due to higher mean capital elasticities at the industry level, and not due to changes in the covariance between final-use shares and industry elasticities. 

\section{The composition of capital and output}
Up to this point I have been working with data that covers all industries, including those for which value-added and capital costs may be particularly hard to measure correctly (e.g. government, owner-occupied housing). The baseline also treats capital as a single aggregate, when the BEA allows for a breakdown into structures, equipment, and intellectual property. The baseline also takes as given the capitalization of intellectual property in the national accounts, which is a recent adjustment that impacts value-added and capital stocks within industries.

\subsection{Private business sector}
To see the influence of the inclusion of government (both general government and government enterprises at federal and state/local levels) and owner-occupied housing on the aggregate elasticities, I remove them both from the calculation of elasticities. In practice, this means deleting their rows and columns from the IO matrix $\Lambda$ as well as their entries from the vector of final-use shares in $\Gamma$.\footnote{In practice there are several industries that are deleted, depending on the year. NAICS includes entries for federal general government (defense), federal general government (non-defense), federal government enterprises, state and local general government, and state and local government enterprises. Prior to 1997, the federal general government categories are combined into a single industry. With respect to housing, both housing and other real estate are excluded. Prior to 1997, those two industries were aggregated into a single real estate industry.} This makes the coverage of the calculation equivalent to the ``Private business sector'' coverage that the BLS uses.\footnote{One could also examine the ``Non-farm private business sector'' by eliminating the industries for farming and forestry, fishing, and other agricultural activities. In practice eliminating those industries does not change the elasticity by an appreciable amount compared to the ``Private business sector''.} 

Figure \ref{FIG_excl_gov_housing} plots the estimated value of the capital elasticity bounds (the no-profit and depreciation cost assumptions) for the private business sector as dark black lines, and the plausible range for the capital elasticity is shaded in dark gray. For comparison, the original bounds using all industries are plotted using the dashed lines, and that range is shaded in light gray. 

As can be seen, the range of the capital elasticity for the private business sector lies everywhere lower than the range for the aggregate economy. In general, both bounds are shifted down by approximately 0.06. Notably, the capital elasticity for the private business sector lies definitively under 1/3 throughout the period, and only during the years 2010-2018 does the upper bound reach that value. The naive value is too high if one is considering just private business sector activity.  

In Table \ref{TAB_scenario}, Panel B, summary statistics are reported for the private business sector alone. Comparing Panel B to Panel A, the downward shift of about 0.06 in the aggregate capital elasticity shows up regardless of how capital costs are calculated. This has the implication that the aggregate labor elasticity is estimated to be higher by about 0.06 in the private business sector. 

Mechanically, the lower elasticity in the private business sector comes almost entirely from the fact that the housing industry has a very high capital cost share under any set of assumptions. The average capital cost share is 0.94 under the no-profit assumption, 0.89 under the investment cost assumption, 0.83 under the user cost assumption, and 0.80 under the depreciation-only assumption. Full statistics on these shares are available in the Appendix. The estimates of $\epsilon_{Kt}$ are weighted averages of cost shares across different industries (with the weights depending on input-output linkages), so the exclusion of owner-occupied housing lowers $\epsilon_{Kt}$ for the private business sector. Government cost shares are similar to the private business sector, and hence excluding government by itself does not alter $\epsilon_{Kt}$ by an appreciable amount.\footnote{Government capital costs are measured differently than for the rest of the economy. When the BEA imputes government value-added, it combines measured government labor compensation with government capital depreciation. Thus, by construction the no-profit (value-added minus labor compensation) and depreciation capital costs are identical for the government industries. In this sense the presence of the government in the baseline calculation of $\epsilon_{Kt}$ pushes the bounds closer together.}

If I restrict myself to the private business sector of the economy then the likely size of $\epsilon_{Kt}$ is well below one-third, and even below 0.30. Nevertheless from Table \ref{TAB_scenario}, Panel B, columns (5)-(7) one can see that the elasticity under all assumptions did rise over time, and in amounts very similar to the rise in the elasticity when the entire economy is considered in Panel A. The private business sector has a similar trajectory of $\epsilon_{Kt}$ over time, but is shifted down compared to the overall economy.

\subsection{Elasticities by type of capital}
To this point I have focused on the elasticity of GDP with respect to aggregate capital, $\epsilon_{Kt}$. But the national accounts data include information on three types of capital: structures, equipment, and intellectual property. It is feasible to calculate separate elasticities for each type separately: structures ($\epsilon_{st,t}$), equipment ($\epsilon_{eq,t}$), and intellectual property ($\epsilon_{ip,t}$).

To construct these estimates, one simply has to expand the matrix $\Lambda$ in equation (\ref{EQ_Lambda}) to include separate columns denoting the cost shares of each type of capital for each industry, and ensure that there are rows of zeroes included in $\Lambda$ for each type. All the capital data for the three types is available from the BEA sources mentioned previously. For the depreciation lower bound, the investment cost assumption, and the user cost assumption, the calculations for the separate elasticities are straightforward. 

The only issue arises with the no-profit upper bound. In this case, the \textit{total} cost of capital is calculated from equation (\ref{EQ_noprofit}) by subtracting labor costs from value-added. This does not provide any information on how those implied capital costs are allocated to structures, equipment, and intellectual property. As a baseline, I distribute the total capital cost in the no-profit scenario across capital types in proportion to the amount of investment done in that capital type in the given year.\footnote{An alternative is to allocate the aggregate no-profit capital costs to capital types in proportion to the stock of each capital type in the given year. This creates some issues with respect to intellectual property, as the absolute size of the IP stock prior to 1970 is so small. For those years the implied cost of capital in the no-profit scenario is below the reported cost of IP depreciation or IP investment.} More specifically, for capital type $k \in (st,eq,ip)$ in industry $i$ at time $t$ I calculate

\begin{equation}
    COST_{ikt}^{NoProf} = \left(VALU_{it} - COST_{iLt}\right)\frac{INV_{ikt}}{\sum_{k \in (st,eq,ip)} INV_{ikt}}.
\end{equation}

Figure \ref{FIG_cap_types} plots the estimates of $\epsilon_{st,t}$, $\epsilon_{eq,t}$, and $\epsilon_{ip,t}$ separately, each with a no-profit upper bound a depreciation lower bound, and the estimates based on investment costs included. I did not plot the user cost series to keep the Figure clear. In the Figure, the bounds for structures appear to be somewhat stable, with a range of about \basestdepr-\basetnoprofit \ throughout the time period, although the lower bound does appear to rise to around 0.10 between 2000 and 2009. For equipment, there is a similar stability in the bounds, and the estimated elasticity appears to be in the range of \baseeqdepr-\baseeqnoprofit \ from 1948-2018. Intellectual property displays a very narrow range from 1948-2000, and that range expands after 2000. Moreover, the range of elasticities for intellectual property climbs from \baseipearlydepr-\baseipearlynoprofit \ around 1948 to \baseiplatedepr-\baseiplatenoprofit \ by 2018.

By construction, the aggregate elasticity $\epsilon_{Kt}$ is the sum of the three individual elasticities in any given year. Hence Figure \ref{FIG_cap_types} provides information on what drove the change in the boundaries of $\epsilon_{Kt}$ over time; it would appear that the increased importance of intellectual property in production was responsible for the shift up.

Table \ref{TAB_type} gives the summary statistics for the three types of capital, under each possible assumption regarding their individual costs. The average elasticity for structures is \basetnoprofit \ under the no-profit assumption, and \basestdepr \ under the depreciation assumption. There is little evidence that the no-profit upper bound increased over time, while the depreciation cost lower bound appears to have risen by about 0.047. An important caveat to the no-profit upper bound numbers here are that they rely on the assumption that capital costs are allocated to structures based on the amount of investment in structures compared to other capital types, as explained above. Thus the no-profit upper bound may itself not be entirely correct (and that caveat holds for all three types of capital).

In Panel B the equipment elasticity averages \baseeqnoprofit \ under the no-profit assumption and \baseeqdepr \ in the depreciation cost assumption. In both cases there is little evidence of an increase in the bounds over time, and in fact the no-profit upper bound declined by 0.015 over the 70 years. Finally, the intellectual property elasticity had an average no-profit upper bound of \baseipnoprofit \ and a depreciation cost lower bound of \baseipdepr. However, in both cases those bounds increased from 1948-2018, the upper bound by 0.076, and the lower bound by 0.052. The investment cost and user cost estimates for intellectual property also increased over time by similar amounts. 

\subsection{De-capitalizing intellectual property}
The prior sub-section showed that an increased importance of intellectual property in production seems largely responsible for the observed increase in the bounds on the aggregate elasticity $\epsilon_{Kt}$. This possibility is consistent with the findings in \cite{ksz2020}, who showed that the revision to the national accounts to capitalize intellectual property, begun by the BEA with their 11th revision in 1999, can explain essentially all of the reported decline in labor's share of GDP. While I am concerned here with the elasticity of GDP with respect to capital (and labor), the same features of the national accounts that \cite{ksz2020} identified may be relevant here. 

In particular, in capitalizing intellectual property (as opposed to treating it as an expense) the BEA revised up the value-added of each industry by an amount equal to the sum of own-account and purchased intellectual property. This also revised gross operating surplus by incorporating own-account intellectual property and revised total depreciation to include that of intellectual property. Following \cite{ksz2020}, I reverse these modifications to strip out the capitalization of intellectual property and then estimate $\epsilon_{Kt}$ again. Details of the modifications to the national accounts data required are in the Appendix.

De-capitalizing IP changes the estimated aggregate elasticities. Figure \ref{FIG_noip_comparison} plots in the dark lines the upper (no-profit) and lower (depreciation-only) bounds for the elasticity $\epsilon_{Kt}$ when intellectual property is de-capitalized from the national accounts. For comparison purposes, the dashed lines plot the upper (no-profit) and lower (depreciation-only) bounds under the baseline situation where IP is considered a capital good. 

As can be seen, there is a distinct shift down in the range of plausible $\epsilon_{Kt}$ values when IP is de-capitalized. The upper bound is well below 1/3 throughout most of the time period, and only rises above it in 2005-2018, and even then the difference is small. There is a similar story for the lower bound, which starts similar to the baseline in 1948, but remains much lower through 2018. Figure \ref{FIG_noip_comparison} indicates that an important part of the apparent rise in $\epsilon_{Kt}$ over time was the capitalization of IP. 

In Table \ref{TAB_scenario}, Panel C, I show the summary statistics when IP is de-capitalized. The mean and median values are lower under all assumptions compared to the baseline, by about 0.03. Perhaps more interesting is columns (5)-(7), which show that the implied rise in $\epsilon_{Kt}$ over time was more muted when IP is de-capitalized, but it does not disappear completely. 

The mitigation of the upward trend is consistent with the findings of \cite{ksz2020} on the mitigation of the \textit{downward} trend in labor's share of GDP. In both cases de-capitalizing IP leaves the size of labor compensation the same, but lowers the size of value-added in each industry. In my case, this implies that there is less value-added ``left over'' to be attributed to capital costs, capital cost shares are lower across industries, and hence the size of $\epsilon_{Kt}$ is lower. 

Panel D of Table \ref{TAB_scenario} shows summary statistics when IP is de-capitalized in the private business sector. The combined effect is to push the no-profit upper bound down lower, to an average of \exclnoprofit \ for $\epsilon_{Kt}$, with a maximum of \exclmaxnoprofit, below 1/3. The depreciation lower bound averages only \excldepr, with a maximum of \exclmaxdepr. With this narrow definition of economic output and excluding IP capital, it is plausible that the capital elasticity was below 0.3 on a regular basis.

\section{An application to growth accounting}
The bounds I have found for the capital and labor elasticities are relevant to the calculation of total factor productivity (TFP) growth. In particular, common series on TFP growth assume that the labor elasticity can be estimated from labor's share of GDP, and that capital's elasticity is one minus the labor elasticity. These elasticities correspond to the ``no-profit'' scenario I use, and are only applicable if the economy has zero markups or market power. This assumption thus provides a bound on TFP growth over time, but may not reflect actual TFP growth. Here I show TFP growth over time when different assumptions about the elasticities are used. 

The baseline calculation I am working with is as follows,

\begin{equation}
	d \ln TFP^s_t = d \ln Y_t - \epsilon^s_{Kt} d \ln K_t - \epsilon^s_{Lt} d \ln L_t. \label{EQ_accounting}
\end{equation}

The difference in log TFP at time $t$, $d \ln TFP^s_t$, depends on the difference in log output, $d \ln Y_t$ minus the effects that are accounted for by capital growth, $d \ln K_t$, and growth in labor inputs, $d \ln L_t$. The data on output and inputs I take directly from the BLS. Growth in the labor input is made up of two parts, the growth rate of hours and the growth rate of labor quality, the latter of which is imputed by the BLS from the composition of the workforce and relative wages. Capital growth is measured as the growth in capital services, also imputed by the BLS.\footnote{To impute capital service the BLS allocates total capital costs across different types of capital. An implicit assumption in that imputation is that total capital costs are equal to all non-labor value-added, or that there are no profits. I take the BLS capital numbers as given, and only focus on the effect of changing the elasticities in equation (\ref{EQ_accounting}).}

The superscript $s$ in equation (\ref{EQ_accounting}) refers to the assumption used to calculate the elasticities $\epsilon^s_{Kt}$ and $\epsilon^s_{Lt}$. I calculate TFP growth for four different values of $s$ corresponding to series I described above in Section 4: no-profit, depreciation only, investment costs, and user costs. By construction, the series of TFP growth calculated using the values of elasticities under the no-profit assumption matches the standard BLS exactly (with minor rounding errors). The other assumptions $s$ yield different series for $d \ln TFP^s_t$.

It is not obvious ex ante whether the growth rates of TFP will be higher or lower than the BLS baseline when I use different assumptions for the elasticities. Overall, the other assumptions give lower values for the capital elasticity, and higher values for the labor elasticity. Whether this leads to higher or lower estimates of TFP growth depends on the relative size of capital growth and labor growth. To the extent that capital growth is \textit{higher} than labor growth, this will tend to lead to \textit{higher} estimates of $d \ln TFP^s_t$ as the elasticities will reduce the implied role of input growth. 

Figure \ref{FIG_tfp_comparison} shows the results of the exercise. I've converted the growth rates of TFP into a level series with 1947 equal to 100, to aid in visualization. What becomes clear is that the no-profit assumption behind the BLS elasticities gives a \textit{lower} bound to the level of TFP over time. Elasticities based on depreciation costs yield an \textit{upper} bound to the TFP series, while the investment cost and user cost elasticities give series lying in between, consistent with the fact that those elasticities also tend to lie in between the bounds. 

In terms of implication, it is likely that the standard BLS numbers are under-stating the size of TFP over time, and therefore the absolute growth rate of TFP. By 2018 under the BLS no-profit assumption TFP is 2.5 times higher than in 1947, but using the depreciation cost elasticities TFP is more than 3 times higher. TFP growth under the BLS baseline average 1.29\% per year, but could be as high as 1.60\% per year given the depreciation only elasticities. Moreover, the no-profit and depreciation cost series are bounds, and it is not necessary that TFP simply followed one of these paths throughout time. In particular, if market power increased over time, as recent research suggests is plausible from 1980 forward, then the actual level of TFP would be moving upward within the bounds (i.e. moving up from the no-profit lower bound) and then the actual growth rate of TFP would be even \textit{higher} than the growth rate implied by the bounds themselves. 

Table \ref{TAB_tfp_scenario} shows the average annual growth rate of TFP, by decade, under different scenarios. Column (1) uses the BLS no-profit scenario, and shows the standard pattern of TFP growth over time (e.g. slowdown in the 2000s), and the overall average growth rate of 1.32\% for 1948-2018. In columns (2)-(4) are the average annual growth rates calculated under the other scenarios for capital costs, and shows that these imply higher growth rates across all decades, but that the variation between decades is preserved. There is still a slowdown starting in the mid-1970s to the mid-1980s, followed by rapid growth in the 1990s, and then a slowdown in growth starting in the early 2000s.

\section{Conclusion}
The elasticities of GDP with respect to capital and labor are central parameters to almost any model of the economy. Values for these elasticities have traditionally been derived from factor share information, leading to the rule of thumb that the capital elasticity is equal to one-third and the labor elasticity two-thirds. 

That rule of thumb requires several strong assumptions, including the existence of an aggregate production function and zero economic profits. In this paper I applied the theory of \cite{bfshortnote,bfprodge} to the calculation of these elasticities, which eliminates those strong assumptions, and allows me to estimate the aggregate elasticities using industry-level data on costs of capital and labor. 

Because of the standard problem of finding capital costs from national accounts data, I create bounds on the elasticities based on different assumptions. An upper bound for the capital elasticity is created by assuming there are zero economic profits, and a lower bound by assuming that depreciation is the only cost of capital. Those bounds indicate a value of the capital elasticity that was \baseearlydepr-\baseearlynoprofit \ from 1948-1995 in the U.S., and \baselatedepr-\baselatenoprofit \ from 1996-2018. If I limit the scope of the economy to the private business sector, or de-capitalize intellectual property from the national accounts, those bounds are shifted down by between 0.03-0.07 in each year. Most of the increase in the bounds after 1995 appears to be due to an increased elasticity with respect to intellectual property capital, while the elasticities with respect to structures and equipment remained stable throughout the period 1948-2018.

The results suggest that the rule of thumb ``alpha equal one-third'' is likely to overstate the size of the capital elasticity, at least for much of the time frame considered, and in particular in the presence of market power. This has consequences for things such as growth accounting. By using the bounds I calculate for the capital elasticity, I show that common BLS estimates of the TFP growth rate and level are likely understated.

While the overall results would suggest that the rule of thumb overstates the capital elasticity (and understates the labor elasticity), it is not wildly inaccurate. Going forward, papers that require an estimate of the capital and labor elasticities could use the boundaries I have calculated as part of robustness and sensitivity checks to confirm their results are not due to the specific elasticities chosen. More generally, studies relying on aggregate elasticities as part of calibration or imputations of productivity would be advised to consider the range of values estimated here to ensure they are not basing their findings on extreme values.

\newpage

\clearpage

\onehalfspacing
%\renewcommand{\refname}{\textbf{REFERENCES}}
%\setlength{\bibsep}{1pt}
{\small
\bibliographystyle{aea}
\bibliography{Elasticity.bib}
}

\clearpage

\begin{figure}[!htb]
\begin{center}
\caption{Boundaries for the aggregate capital elasticity, $\epsilon_{Kt}$, U.S. 1948-2018}
\label{FIG_cap_comparison}
\includegraphics[width=1.0\textwidth]{fig_cap_base_comparison.eps}
\end{center}
\vspace{-.5cm}\singlespacing {\footnotesize \textbf{Notes}: The estimate of the aggregate capital elasticity, $\epsilon_K$, is made using equation (\ref{EQ_E}) under various assumptions explained in detail in the text. The no-profit assumption assumes capital costs equal all value-added minus labor compensation. The depreciation-only assumption assumes capital costs equal the value of depreciation reported. The primary data source for all estimates is the Bureau of Economic Analysis, with input-output tables, capital stocks by industry, compensation by industry, and value-added by industry using different industrial classifications merged according to a methodology described in the Appendix.
}
\end{figure}

\clearpage


\begin{figure}[!htb]
\begin{center}
\caption{Alternative estimates of the aggregate capital elasticity, $\epsilon_{Kt}$, U.S. 1948-2018}
\label{FIG_all_comparison}
\includegraphics[width=1.0\textwidth]{fig_cap_all_comparison.eps}
\end{center}
\vspace{-.5cm}\singlespacing {\footnotesize \textbf{Notes}: The estimate of the aggregate capital elasticity, $\epsilon_K$, is made using equation (\ref{EQ_E}) under various assumptions explained in the text. The no-profit and depreciation-only bounds are the same as in Figure \ref{FIG_cap_comparison}. The investment cost assumption assumes capital costs equal reported investment, and the user cost assumption assumes capital costs are determined by a standard user cost formula from \cite{halljorg1967}. The primary data source for all estimates is the Bureau of Economic Analysis, with input-output tables, capital stocks by industry, compensation by industry, and value-added by industry using different industrial classifications merged according to a methodology described in the Appendix. Additional information on nominal rates of return and inflations rates used for the user cost calculation is from the Federal Reserve. 
}
\end{figure}

\clearpage


\begin{figure}[!htb]
\begin{center}
\caption{Comparison of estimated elasticity, $\epsilon_{Kt}$, to cost ratio, $s^{Cost}_{Kt}$}
\label{FIG_ratio_comparison}
\includegraphics[width=1.0\textwidth]{fig_cap_total_comparison.eps}
\end{center}
\vspace{-.5cm}\singlespacing {\footnotesize \textbf{Notes}: Estimate of the aggregate capital elasticity, $\epsilon_{Kt}$, are plotted on the y-axis and made using equation (\ref{EQ_E}) under various assumptions explained in the text and denoted in the legend. The cost ratio of capital, $s^{Cost}_{Kt}$, is calculated as in equation (\ref{EQ_scost}) under the same assumptions. The position of the $\epsilon_{Kt}$ estimates above the 45-degree line indicates that the input/output relationships matter to some degree, and that industries with high capital elasticities tend to be suppliers to other industries, pulling up the aggregate elasticity.
}
\end{figure}

\clearpage


\begin{figure}[!htb]
\begin{center}
\caption{Aggregate capital elasticity and mean industry-level capital elasticity, U.S. 1948-2018}
\label{FIG_op_comparison}
\includegraphics[width=1.0\textwidth]{fig_cap_op_comparison.eps}
\end{center}
\vspace{-.5cm}\singlespacing {\footnotesize \textbf{Notes}: The estimates of the aggregate capital elasticity (black lines), $\epsilon_{Kt}$, are made using equation (\ref{EQ_E}) in the text. The two estimates differ in the assumption regarding capital costs - depreciation costs only or a no-profit assumption - as explained in the text. The mean industry-level capital elasticity (gray lines) is the term $\overline{\ell}_{Kt}$ from equation (\ref{EQ_op}) . It is the raw average of the elements $\ell_{iKt}$ from the Leontief inverse found in equation (\ref{EQ_E}). The difference between the aggregate elasticity and the mean industry-level elasticity is due to the covariance of the final-use share of an industry and the industry-level elasticity. In both the depreciation and no-profit case, the covariances are positive as the aggregate elasticity lies above the mean industry-level elasticity. The figure shows the trend upward in the aggregate elasticity bounds was due to industries, on average, having higher capital elasticities over time, and not due to a change in the covariance of industry size (in final-use terms) and the size of the elasticity.

}
\end{figure}

\clearpage


\begin{figure}[!htb]
\begin{center}
\caption{Implied aggregate gross output markups, $\mu^{GO}_t$, by capital cost assumption}
\label{FIG_markup}
\includegraphics[width=1.0\textwidth]{fig_cap_markupgo_comparison.eps}
\end{center}
\vspace{-.5cm}\singlespacing {\footnotesize \textbf{Notes}: The estimated markups are calculated using equation (\ref{EQ_markup}). The three series (no-profit, depreciation, investment cost) refer to the capital cost assumption used to calculate the industry-level markups that are used to calculate the aggregate markup.
}
\end{figure}

\clearpage
\begin{figure}[!htb]
\begin{center}
\caption{Estimates of aggregate capital elasticity, private business sector, U.S. 1948-2018}
\label{FIG_excl_gov_housing}
\includegraphics[width=1.0\textwidth]{fig_cap_priv_comparison.eps}
\end{center}
\vspace{-.5cm}\singlespacing {\footnotesize \textbf{Notes}: The estimate of the aggregate capital elasticity, $\epsilon_{Kt}$, is made using equation (\ref{EQ_E}) in the text. Dashed lines refer to the upper (no-profit) and lower (depreciation-only) bounds of $\epsilon_{Kt}$ calculated including all industries. The dark lines refer to the upper (no-profit) and lower (depreciation-only) bounds of $\epsilon_{Kt}$ calculated for the private business sector (e.g. excluding owner-occupied housing and government). 
}
\end{figure}

\clearpage

\begin{figure}[!htb]
\begin{center}
\caption{Estimates of capital elasticity, by type of capital}
\label{FIG_cap_types}
\includegraphics[width=1.0\textwidth]{fig_cap_combined_comparison.eps}
\end{center}
\vspace{-.5cm}\singlespacing {\footnotesize \textbf{Notes}: The estimates of the capital elasticities for structures ($\epsilon_{st,t}$), equipment ($\epsilon_{eq,t}$), and intellectual property ($\epsilon_{ip,t}$), are made using equation (\ref{EQ_E}) in the text. For each type of capital three estimates are shown, based on assumed cost of capital: no-profit assumption, depreciation cost only, investment costs. See text for details of the three assumptions.
}
\end{figure}

\clearpage

\begin{figure}[!htb]
\begin{center}
\caption{Estimates of aggregate capital elasticity, de-capitalizing IP, U.S. 1948-2018}
\label{FIG_noip_comparison}
\includegraphics[width=1.0\textwidth]{fig_cap_noip_comparison.eps}
\end{center}
\vspace{-.5cm}\singlespacing {\footnotesize \textbf{Notes}: The estimate of the aggregate capital elasticity, $\epsilon_{Kt}$, is made using equation (\ref{EQ_E}) in the text. Dashed lines refer to the baseline upper (no-profit) and lower (depreciation-only) bounds of $\epsilon_{Kt}$ calculated with IP included as a capital good. The dark lines refer to the upper (no-profit) and lower (depreciation-only) bounds of $\epsilon_{Kt}$ calculated when IP is de-capitalized from the national accounts as described in the text.
}
\end{figure}

\clearpage

\begin{figure}[!htb]
\begin{center}
\caption{TFP levels under different elasticity assumptions, U.S. 1948-2018}
\label{FIG_tfp_comparison}
\includegraphics[width=1.0\textwidth]{fig_tfp_comparison.eps}
\end{center}
\vspace{-.5cm}\singlespacing {\footnotesize \textbf{Notes}: The level of TFP is calculated using equation (\ref{EQ_accounting}) which follows the methodology of the BLS. Data on output, capital services, and labor inputs are from \cite{fernald2014}. The four series plotted differ in the elasticities with respect to capital and labor used in the calculation. Those four series correspond to the four different methods used in this paper to calculate the elasticities, and are based on different assumptions regarding how to calculate the cost of capital by industry. My calculations exclude government and owner-occupied housing, to match the BLS calculations for the private business sector. The ``no-profit'' elasticities I calculate match the elasticities used by the BLS, and so that series matches the BLS multi-factor productivity numbers with only minor rounding errors. 
}
\end{figure}

\clearpage

\begin{table}[!htb]
\begin{center}
\label{TAB_series}
\caption{Industrial classification of data by year}
\begin{tabular}{lrrrr}
\midrule
	    &             & Value-added   & \\
Series  & I/O tables  & components & Capital stock \\ 
\midrule
1948-62 & NAICS 2012 (47 ind) & SIC 1972 & BEA/NAICS 2012 \\
1963-86 & NAICS 2012 (65 ind) & SIC 1972 & BEA/NAICS 2012 \\
1987-96 & NAICS 2012 (65 ind) & SIC 1987 & BEA/NAICS 2012 \\
1997-2018 & NAICS 2012 (71 ind) & NAICS 2012 & BEA/NAICS 2012 \\ 
\midrule
\end{tabular}
\end{center}
{\footnotesize Notes: This table shows the classifications used for each range of years. The complete mapping of industry data across sources is provided in the Appendix. All data are from the BEA, and described in detail in Section 3.}
\end{table}

\clearpage

\begin{table}[!htb]
\begin{center}
\caption{Estimates of U.S. capital elasticity, $\epsilon_K$, under different assumptions}
\label{TAB_scenario}
{\footnotesize
\begin{tabularx}{\textwidth}{lXXXXXXX}
\midrule
        & \multicolumn{4}{c}{Summary statistics, $\epsilon_{Kt}$, 1948-2018:}  & \multicolumn{3}{c}{Fitted change 1948-2018:} \\ \cmidrule(lr){2-5} \cmidrule(lr){6-8}
 &  Mean & Median  & Minimum & Maximum  & $\Delta \hat{\epsilon}_{K,48-18}$ & Slope ($\hat{\beta}_1$) & R-squared \\
Assumption & (1) & (2) & (3) & (4) & (5) & (6) & (7) \\
\midrule
\input{tab_scenario_summary.txt}
\midrule
\end{tabularx}
}
\end{center}
\vspace{-.5cm}\singlespacing {\footnotesize \textbf{Notes}: The calculation of $\epsilon_{Kt}$ is described in the text. The panels of the table refer to different assumptions made regarding the inclusion/exclusion of owner-occupied housing and intellectual property capital (the baseline includes both). Within the panel, the rows refer to assumptions about the capital costs by industry, $COST_{iKt}$, that are used to calculate $\epsilon_{Kt}$. The specifics of those assumptions are discussed in the text. The fitted change is estimated from a simple OLS regression of $\epsilon_{Kt}$ against time for the given variant, with $\hat{\beta}_1$ showing the estimated change per year, and $\Delta \hat{\epsilon}_{K,48-18} = 70\times\hat{\beta}_1$ being the estimated overall change from 1948 to 2018. The R-squared is from the simple OLS regression.
}
\end{table}

\clearpage

\begin{table}[!htb]
\begin{center}
\caption{Capital costs as share of factor costs and value-added, by sector}
\label{TAB_ratios}
{\footnotesize
\begin{tabularx}{\textwidth}{lXXXXXXXX}
\midrule
        & \multicolumn{8}{c}{Summary statistics, 1948-2018:} \\ \cmidrule(lr){2-9} 
        & \multicolumn{4}{c}{Capital costs/Factor costs, $s^{Cost}_{Kt}$} & \multicolumn{4}{c}{Capital costs/Value-added, $s^{VA}_{Kt}$} \\ \cmidrule(lr){2-5} \cmidrule(lr){6-9} 
 &  Mean & Median  & Minimum & Maximum  &  Mean & Median  & Minimum & Maximum \\
Variant & (1) & (2) & (3) & (4) & (5) & (6) & (7) & (8) \\
\midrule
\input{tab_cost_summary.txt}
\midrule
\end{tabularx}
}
\end{center}
\vspace{-.5cm}\singlespacing {\footnotesize \textbf{Notes}: The cost ratios are calculated as in equations (\ref{EQ_scost}) and (\ref{EQ_sva}). The panels of the table refer to different sectors of the economy. Panel A includes all industries. Panel B is just the private business sector, which excludes owner-occupied housing and government. Owner-occupied housing refers to NAICS codes HS, ORE, and 531. Government refers to NAICS codes GFGD, GFGN, GFE, GSLG, GSLE, and GFG, which covers federal, state, and local government, both general and enterprises. In each row, the assumption made to calculate capital costs is labeled, as described in the text. Columns (1)-(4) are summary statistics over 1948-2018 for the total estimated capital costs divided by total factor costs (the sum of capital costs and labor costs). Columns (5)-(9) are summary statistics over 1948-2018 for total capital costs divided by value-added. 
}
\end{table}

\clearpage

\begin{table}[!htb]
\begin{center}
\caption{Estimates of U.S. capital elasticity, by capital type}
\label{TAB_type}
{\footnotesize
\begin{tabularx}{\textwidth}{lXXXXXXX}
\midrule
        & \multicolumn{4}{c}{Summary statistics, $\epsilon_{it}$, 1948-2018:}  & \multicolumn{3}{c}{Fitted change 1948-2018:} \\ \cmidrule(lr){2-5} \cmidrule(lr){6-8}
 &  Mean & Median  & Minimum & Maximum  & $\Delta \hat{\epsilon}_{i,48-18}$ & Slope ($\hat{\beta}_1$) & R-squared \\
Variant & (1) & (2) & (3) & (4) & (5) & (6) & (7) \\
\midrule
\input{tab_capital_summary.txt}
\midrule
\end{tabularx}
}
\end{center}
\vspace{-.5cm}\singlespacing {\footnotesize \textbf{Notes}: The calculation of $\epsilon_{it}$ with $i \in (st,eq,ip)$ is described in the text. The panels of the table differ in the type of capital (structures, equipment, intellectual property) the elasticity is calculated for. Within each panel, the no-profit variation splits the total capital cost across the three capital types according to the amount of investment spending on that capital type in a given year. User cost, investment cost, and depreciation cost variants use costs of capital for that type calculated directly, according to methods described in the text. The fitted change is estimated from a simple OLS regression of $\epsilon_{it}$ against time for the given variant, with $\hat{\beta}_1$ showing the estimated change per year, and $\Delta \hat{\epsilon}_{i,48-18} = 70\times\hat{\beta}_1$ being the estimated overall change from 1948 to 2018. The R-squared is from the simple OLS regression.
}
\end{table}

\clearpage

\begin{table}[!htb]
\begin{center}
\caption{Average annual TFP growth (\%), by capital cost assumption}
\label{TAB_tfp_scenario}
\begin{tabular}{lcccc}
\midrule
        & \multicolumn{4}{c}{Assumption on capital costs:} \\ \cmidrule(lr){2-5}
 &  No-profit (BLS) & User cost & Investment cost & Depreciation cost \\
Years & (1) & (2) & (3) & (4) \\
\midrule
\input{tab_tfp_scenario.txt}
\midrule
\end{tabular}
\end{center}
\vspace{-.5cm}\singlespacing {\footnotesize \textbf{Notes}: All growth rates reported in percents. TFP growth is calculated using equation (\ref{EQ_accounting}) to find annual growth rate, which is then converted to a level of TFP using 1948 as a base = 100. Average annual growth rates are calculated using those levels of TFP. This is done for the private business sector only (excluding government and housing) to match BLS procedures. The different assumptions on capital costs correspond to the $s$ parameter in equation (\ref{EQ_accounting}) and refer to different assumptions about capital costs used to calculate $\epsilon_{Kt}^s$ and $\epsilon_{Lt}^s$. The ``no-profit'' capital cost assumption in column (1) is equivalent to the BLS assumption regarding elasticities.
}
\end{table}
\end{document}