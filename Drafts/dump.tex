In Figure \ref{FIG_cost_comparison} I plot the estimated values of $\epsilon_{Kt}$ in each year, under each of the four main assumptions regarding capital cost, on the y-axis. On the x-axis is the value of the ratio $s^{Cost}_{Kt}$ from the same year, under the same assumption regarding capital costs. As one can see, the relationship is tight under any assumption regarding capital costs, consistent with the comparison of Tables \ref{TAB_scenario} and \ref{TAB_ratios}. 

The relationship is not only tight, it is almost identical to the 45-degree line, implying that the  aggregate ratio $s^{Cost}_{Kt}$ ignoring input/output relationships is very close to the value of $\epsilon_{Kt}$ that incorporates those input/output relationships. To make this point more clearly, I run univariate regressions of the form

\begin{equation}
	\epsilon_{Kt} = \beta_0 + \beta_1 s^{Cost}_{Kt} + \varepsilon
\end{equation}

for a given assumption regarding capital costs. 

Table \ref{TAB_regression} reports the results of these regressions in Panel A. Under the no-profit assumption about capital costs the results for $\hat{\beta}_0$ indicate that $\epsilon_{Kt}$ is about 0.018 higher than $s^{Cost}_{Kt}$ on average, consistent with the eyeball comparison of Tables \ref{TAB_scenario} and \ref{TAB_ratios}. The $\hat{\beta}_1$ value of 0.990 indicates that the gap between $\epsilon_{Kt}$ and $s^{Cost}_{Kt}$ is similar no matter the size of $s^{Cost}_{Kt}$.

Reading over to columns (2)-(3) of the table, the relationships of the cost ratios to the estimated elasticities change. Using either depreciation or investment to determine capital costs, the elasticities are slightly \textit{lower} than the $s^{Cost}_{Kt}$ ratios, although the differences are small. Perhaps more relevant, as $s^{Cost}_{Kt}$ increases, the estimated $\epsilon_{Kt}$ increases more than one-for-one (i.e. the $\hat{\beta}_1$ values are 1.058 and 1.061). Hence as the cost share of capital increases the deviation of the estimated elasticity from that cost share increases as well. 

Column (4) is based on the user cost of capital, and here the values of $s^{Cost}_{Kt}$ and $\epsilon_{Kt}$ are quite close, with $\hat{\beta}_0$ equal to 0.006 and $\hat{\beta}_1$ equal to 1.001. It is important to note that this close relationships does \textit{not} constitute proof that the user cost of capital is the correct way to value capital costs. It would be correct to say that \textit{if} one believed that user cost was the correct way to determine capital costs, \textit{then} one could safely ignore input/output relationships in calculating $\epsilon_{Kt}$, and rely directly on the $s^{Cost}_{Kt}$ ratio. 

The final column lumps together all four different assumptions used into a single regression, thus matching the fitted line shown in Figure \ref{FIG_cost_comparison}. The results here are similar to those with depreciation or investment costs, as the level of $\epsilon_{Kt}$ is captured well by $s^{Cost}_{Kt}$, but their values deviate as the cost ratio increases.

Overall, the results in Table \ref{TAB_regression} suggest that an aggregate ratio $s^{Cost}_{Kt}$ is a decent approximation to the more sophisticated $\epsilon_{Kt}$. Even with the deprecation or investment cost assumption, the deviation of $\epsilon_{Kt}$ from the cost ratio remains small, as indicated by the very high R-squared across all regressions. The input/output structure of industries is theoretically the correct way to find $\epsilon_{Kt}$, but in practice one could substitute an aggregate ratio $s^{Cost}_{Kt}$ without too much loss of information. This may be useful in situations where full input/output tables do not exist. 

At the risk of belaboring a point, I will mention again that nothing in this sub-section tells us which assumption about capital costs is the correct one to use. The close relationships of the ratios to the estimated elasticity simply tells us that ignoring input/output relationships may not be fatal.