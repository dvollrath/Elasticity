\documentclass[11pt]{article}
%%%%%%%%%%%%%%%%%%%%%%%%%%%%%%%%%%%%%%%%
\usepackage{amsmath}
\usepackage{verbatim}
\usepackage[usenames,dvipsnames]{color}
\usepackage{setspace}
\usepackage{lscape}
\usepackage{longtable}
\usepackage[top=1.25in,bottom=1.25in,left=1in,right=1in]{geometry}
\usepackage{graphicx}
\usepackage{epstopdf}
\usepackage{epsfig}
\usepackage{fancyhdr}
\usepackage{booktabs}
\usepackage{dcolumn}
\usepackage{arydshln}
\usepackage{natbib}
\usepackage{tabularx}
\usepackage{subfigure}
\usepackage{hyperref}
\usepackage{xcolor}

\hypersetup{
    colorlinks,
    linkcolor={red!50!black},
    citecolor={blue!50!black},
    urlcolor={blue!80!black}
}

\newtheorem{proposition}{Proposition}
\newtheorem{corollary}{Corollary}

\renewcommand{\thetable}{A.\arabic{table}}
\renewcommand{\thesection}{A.\arabic{section}}
\renewcommand{\theequation}{A.\arabic{equation}}
\makeatletter
\renewcommand{\l@section}{\@dottedtocline{1}{1.5em}{2.6em}}
\renewcommand{\l@subsection}{\@dottedtocline{2}{4.0em}{3.6em}}
\renewcommand{\l@subsubsection}{\@dottedtocline{3}{7.4em}{4.5em}}
\makeatother

\setcounter{MaxMatrixCols}{10}
\newcolumntype{d}[1]{D{.}{.}{-2.#1}}
\newenvironment{proof}[1][Proof]{\noindent\textbf{#1.} }{\ \rule{0.5em}{0.5em}}
\setlength{\columnsep}{.2in}
%\psset{unit=1cm}
\newcolumntype{R}{>{\raggedleft\arraybackslash}X}

\def\sym#1{\ifmmode^{#1}\else\(^{#1}\)\fi}

\begin{document}
\noindent {\Large \textbf{\strut The Elasticity of Aggregate Output with Respect to Capital and Labor}}


\section*{Response to Referee 2}
\onehalfspacing 

First, my thanks to the referee for what was clearly a close and careful read of the original paper. From my perspective, you had two major points regarding the focus/contribution of the paper and the methodology in calculation. I start with some repsonses on those two major points, and how I addressed them. Following that are responses to the other smaller points. In each case, quotations from your original report are in bold, followed by my response. In several cases I have not cut/paste your entire comment, only showing what I considered the ``heart'' of the question or issue.

\subsection*{Comment 1}
\textbf{The question is then: how different is the “correct” lower bound, calculated from the Baqaee-Farhi approach, from a naive cost share formula that works for a one-sector economy? In a sense, this is the crux of the paper—since aside from this distinction, all results could be obtained using extremely simple and well-known formulas. The paper tackles this question in Figure 3, showing that the correct B-F elasticities are similar to, but mildly higher than, the aggregate cost shares. (On pages 17 and 18, the paper refers to this result as showing that cost shares are a “decent” or “reasonable” approximation of the true elasticity in this case.) I find the explanation of the differences between the two calculations to be a little cryptic.}

\textbf{I think the idea is that capital tends to be used upstream in production, so that the value added from capital passes through more layers of markups before becoming final output, and therefore the aggregate cost share understates capital’s contribution to output after correcting for these markups. But this is not the only possible explanation—the zero-return-after-depreciation assumption will imply heterogeneous markups across sectors, and that heterogeneity could play some role too. I would like to see a much more detailed analysis on these points.}

Your point on the contribution here is well taken. While I think there is value in the paper of explicitly comparing the depreciation cost lower bound (and the other alternatives) to the simple no-profit upper bound, that upper bound is readily calculated and not novel. And hence the point is well taken that a more thorough explanation is required for why the elasticity, $\epsilon_{Kt}$, and factor cost share, $s_{Kt}^{Cost}$, take on the magnitudes that they do, and are related such that $s_{Kt}^{Cost} < \epsilon_{Kt}$ generally in cases where there are economic profits (and $s_{Lt}^{Cost} > \epsilon_{Lt}$). In the new draft I've expanded on this both in the paper itself and in the Appendix. 

Section 5.2 of the paper again reports the comparison of the elasticity to the factor cost share for capital, documenting the basic facts (the Figure 3 you noted in the original report). A new Section 5.3 picks up on that and offers an explanation for their relationship, which as you intuited in your report, has to do with the markups: their size, their heterogeneity, and where they lie in the input/output structure. That section provides evidence on the size of the aggregate gross output markup implied by my calculations of $\epsilon_K$, and compares that to existing estimates to assess whether those markups seem reasonable (short answer, they do). 

The explanation in 5.3 is high-level and intuitive, so in Appendix A.6 I provide a more complete mathematical breakdown of how markups create a wedge between $\epsilon_{Kt}$ and $s_{Kt}^{Cost}$. I've left this detailed explanation in the Appendix at the moment, although based on your report my guess is that you'd see this as more central to the paper. This was a case of trying to appease comments across referees, others of whom saw this as a more subsidiary point. Should I be fortunate enough to move forward with this paper, I'd be open to reconsidering this if the Editor feels the material belongs in a more prominent place. 

Regardless, as I said your intuition on what is driving the wedge between $\epsilon_{Kt}$ and $s_{Kt}^{Cost}$ was correct. I find it useful to see the effect of markups here in two ways. The first is through dispersion in markups across industries, which would create a wedge even if there are no input/output relationships between industries. Disperse markups mean that the allocation of total costs across industries is not the same as the allocation of total expenditure across industries. High-markup industries have costs that are too low compared to expenditures, and low-markup industries have costs that are too high. Combine this with a positive correlation of industry capital intensity (i.e. the share of capital in their factor costs) and markups, and you get one force pushing $s_{Kt}^{Cost}$ below $\epsilon_{Kt}$. Low-markup industries are over-weighted in total costs, and also have low capital intensities, so the overall share of capital in costs is lower than in an efficient economy (e.g. where all markups are equal to one another).

The second way markups matter is through the input/output structure itself. Industry markups are applied at each step through supply chain. This again opens up the possibility that the allocation of costs across industries is not proportional to the allocation of final expenditure across industries. Industries that tend to be ``upstream'', mainly supplying intermediates to other industries rather than final goods, will incur costs that are lower than what would occur in an efficient economy, due to the multiple markups applied along the supply chain. If those upstream industries also tend to be relatively capital intense, then there are relatively few costs being incurred by capital intense industries, and this pulls down the ratio $s_{Kt}^{Cost}$. The larger the markups in the economy, the more severe the multiple marginalization is, and the larger this effect.

In Appendix A.6 I show that the underlying industry-level data is consistent with the two effects of markups. For each industry $j$ I can calculate a gross output markup, $\mu_{jt}$, and capital costs as a share of factor costs, $s_{Kjt}^{Cost}$, given the assumption used for finding capital costs (e.g. depreciation costs). I show that there is a positive correlation between $\mu_j$ and $s_{Kjt}^{Cost}$ in the depreciation cost scenario (the positive correlation exists with investment cost or user cost data as well) for all industry/year observations, even after removing year fixed effects.

Next, I calculate a measure of how ``upstream'' each industry is, $u_{jt}$, as the share of gross output in industry $j$ that is used as intermediate inputs by other industries. Formally I calculate $u_{jt} = 1 - f_{jt}/GO_{jt}$, where $f_{jt}$ is final use of industry $j$ and $GO_{jt}$ is gross output of industry $j$ in a given year. Then I show that there is a positive correlation between $u_{jt}$ and $s_{Kjt}^{Cost}$ in the depreciation cost scenario (again, this holds for other capital cost assumptions), after controlling for year fixed effects.

Quantitatively, while $s_{Kt}^{Cost} < \epsilon_{Kt}$ the difference does not seem that large (between 0.02 and 0.05 normally in the depreciation cost scenario). Markups are disperse, but not dispere enough to generate a larger wedge. Markups are above one, but not far enough above one to generate a larger wedge. There is no sense in which this is ``right'' or ``wrong'', as there is no \textit{a priori} reason to suspect that the wedge should be large or small. It is plausible to ask whether the markups that I do calculate seem reasonable. If those markups look ``small'' then I may be underestimating the wedge (because I'm over-estimating capital costs). If those markups look ``big'', then I may be overestimating the wedge (because I'm under-estimating capital costs). 

As mentioned above, in Section 5.3, Figure 5, I show evidence on the aggregate gross output markup implied by each scenario. For the depreciation cost scenario, this markup drifts down from 1.14 to 1.10 between 1948 and 1980, and then rises to about 1.15 by 2018. Taking the depreciation costs seriously as a bound on capital costs, then the true aggregate gross output markup should lie somewhere between this upper bound and a lower bound of 1 (no profits). In Section 5.3 I compare these markup bounds to other existing estimates of aggregate markups (taking care to distinguish between value-added and gross output markups), and the bounds appear reasonable. 

So in the end the wedge between $\epsilon_{Kt}$ and $s_{Kt}^{Cost}$ is due to markups, combined with some particular relationships between capital intensity and markups, and capital intensity and the position of industries within the input/output structure. The size of that wedge does not appear quantitatively large, but that is consistent with other evidence on the size of markups in the economy.

\textbf{I would also like to be sure that the discrepancies discussed in comment 2 (below) are not playing any role here.} 

See the below point for the full explanation of those discrepancies (and how they are resolved). They do not influence the size or direction of the wedge between $\epsilon_{Kt}$ and $s_{Kt}^{Cost}$. 

\subsection*{Comment 2}
\textbf{Appendix A.4 discusses the calculation for the “upper bound” case where we assume zero economic profit. Since the capital elasticity equals the aggregate cost share in this case, in principle we have two equivalent ways to calculate the elasticity: either do the full B-F analy- sis as for the lower bound, or simply read off cost shares from the aggregate data. Appendix figure 2 shows that the two approaches deliver surprisingly different results...}

\textbf{What particularly worries me here is that whatever is causing this discrepancy might also be playing a role in the lower bound calculation. There, too, the B-F method gives an elasticity a few percentage points higher than an aggregate cost share. Since the two calculations are supposed to differ in the lower bound case, I initially assumed that this difference had some substantive economic content (and I recommend in comment 1 that this be explored much further). But given that there is a similar difference in the upper bound case where, economi- cally, the elasticities should be identical, I lose confidence in interpreting the results.}

You're absolutely correct that the reported discrepancy between $\epsilon_{Kt}$ and $s_{Kt}^{VA}$ in the no-profit scenario does not make sense from a theoretical perspective. And given that discrepancy, it made sense to worry that the wedges between $\epsilon_{Kt}$ and $s_{Kt}^{Cost}$ in the prior point were also due to a computational issue, and not to a real economic phenomenon.

I've identified the reasons that discrepancy arose in the first place, and in the current draft have corrected those. Computationally it is now the case that $\epsilon_{Kt}$ and $s_{Kt}^{VA}$ are identical in the no-profit scenario, and hence there is no lingering worry regarding a computational issue behind the differences in $\epsilon_{Kt}$ and $s_{Kt}^{Cost}$.

One reason for the discrepancy can be discussed and dealt with quickly. There was a classic ``garbage in/garbage out'' coding issue, where incorrect data on final use shares was being pulled in for a set of years. That was fixed.

The other reason for the discrepancy, as you thought might be possible, was due to the underlying input/output matrix I was using and how I treated the Used/Scrap and Noncomparable imports entries. That's the actual larger issue, and resulted in a more significant change to my methodology, and I'll describe that in full detail next. In the end it had a small impact on the results.

\textbf{The explanations given in appendix A.4 are also confusing. The first explanation is that “in trying to invert a 71×71 matrix... the determinant becomes very sensitive to rounding errors”. This is nonsensical to me: there is no way that the Leontief inverse of a reasonable economy should be so ill-conditioned that numerical error is that large. The third explanation is that Baqaee and Farhi’s theory is a first-order approximation and the (elasticity = cost share) result holds in the limit". This also seems nonsensical: everything here is calculated using the same first-order approximation and should agree exactly, up to some trivial numerical error. The only explanation that makes sense to me is the second one, which is that the input-output matrix has an “other” industry that cannot be included in the calculation—which I would broaden to the observation that the theory and input-output matrix may have some tricky inconsistencies (e.g. what about imports and exports?). But this seems like something that can be handled, albeit imperfectly, given sufficient care. To repeat myself, it’s very important that the B-F and share-based calculations are as consistent as possible, since otherwise we have no way of understanding whether the discrepancies (which should be a major theme of the paper) are meaningful.}

So, you're absolutely correct here that the 1st and 3rd explanations for the discrepancy don't make sense. And you're correct that more careful treatment of the underlying input/output tables can solve the issue. The new version of the paper is updated to make that more careful treatment, and the discrepancy is eliminated. 

Details. To recall some terminology from the paper, I need to find the industry-by-industry $COST_{ij}$ terms that make up the matrix $\Lambda$ from equation (9) in the paper. In the original paper you read I was using the BEA's Use Table, Before Redefinitions as the basis for this $COST_{ij}$ information. To your point, the Use Table is a commodity-by-industry table, and there are two more commodities (Used/Scrap and Noncomparable imports) than there are industries. 

What I had been doing was treating Used/Scrap and Noncomparable imports as part of value-added (increasing reported value-added up by the sum of spending on the two while holding gross output constant), rather than as intermediate purchases. Their small size did not change the amounts of value-added by much in each industry. The remaining input/output table was square (the number of commodities matched the number of industries), but not consistent along th two dimensions (gross output of the industries did not match the gross output of the matching commodity). In the original paper this created only small numerical discrepancies (in all scenarios) that were apparent when comparing $\epsilon_{Kt}$ and $s_{Kt}^{VA}$ in the no-profit scenario, where theory gave us the answer that the two should be equal.

Why did I use this ad hoc method? I was after annual results, and the BEA does not provide industry-by-industry input/output tables on an annual basis. The coding error mentioned above sent me off down a path of a computation problem (the matrix invesion stuff), because I couldn't reconcile my modification of the Use Table with the results I was getting. As you could tell, this simply confused the issue. 

What has been the resolution of this issue in the current draft? Given your comments and the ad hoc nature of the original approach, I scrapped it and re-estimated everything by first creating an internally consistent industry-by-industry input/output table for each year.\footnote{Consistent in the sense that gross output of industries is identical whether summed by use (intermediates sold plus final use) or over production (intermediates purchased plus value-added).} I give full details on how I do this in Appendix A.4. The summary is as follows. Each year the BEA reports the ``Before Redefinitions'' Use Table (amounts of commodities used by each industry as inputs) and Make Table (amounts of commodities produced by each industry as outputs). The Use Table tells you how many dollars of each commodity industry $j$ purchased, and the Make Table (scaled by commodity gross output) tells you how many dollars of inputs from industry $i$ are necesary to produce a dollar of commodity $j$. Those can be combined to derive an industry-by-industry table of purchases of inputs from industry $i$ by industry $j$. This is standard I/O algebra.\footnote{For example, see Jiemin Guo, Ann Lawson, and Mark Planting. 2002. ``From Make-Use to Symmetric I-O Tables: An Assessment of Alternative Technology Assumptions.'' Bureau of Economic Analysis. WP2002-03. \url{https://www.bea.gov/system/files/papers/WP2002-3.pdf}}

The industry-by-industry table that is produced is internally consistent, and using this as the basis for calculations results in $\epsilon_{Kt} = s_{Kt}^{VA}$ under the no-profit assumption (to at least the 8 decimal places I looked at), as per theory. 

This approach ensures that the calculations match theory, but it does retain one potential issue. This is the ``Before Redefinitions'' qualification on the Tables. The BEA themselves, in select years, produces industry-by-industry total requirements tables. But those are ``After Redefinitions'' to the Make and Use Table, where they tweak the amount of commodities produced by each industry, or the use of commodities by different industries, based on additional outside information. There is no algorithmic way of recreating these tweaks, which is why the BEA does not produce these ``After Redefintion'' industry-by-industry tables annually (at least prior to 1997). The worry is that my algorithmic creation of ``Before Redefintion'' industry-by-industry tables leads to some systemic bias or distortion in the elasticities, and that I'm trading one discrepancy for another.

The advantage of the new approach is that I can assess the possible size of this bias for a subset of years, and show this in Appendix A.5. The BEA does provide ``After Redefinition'' industry-by-industry tables annually from 1997-2018. I calculate $\epsilon_K$ for each of those years using their ``After Redefintion'' tables, and compare that to the $\epsilon_K$ I get based on my own ``Before Redefinition'' tables. I can do this under each of the different scenarios regarding capital costs (no-profit, deprecation cost, investment cost, and user cost). The results in Appendix A.5 show that, on average, my ``Before Redefinition'' estimates of $\epsilon_K$ are about 0.0024-0.0027 higher than what you get with the ``After Redefinition'' tables. The small size of the these suggests that the redefinitions done by the BEA are not causing me to obtain meaningful differences in the elasticities. Of note, the differences between Before and After Redefinition results hold regardless of the assumption on capital costs.

Given your comments and the small scale of the discrepancy with the ``Before'' versus ``After'' tables, ensuring that the results match the theoretical predictions ($\epsilon_{Kt} = s_{Kt}^{VA}$ in the no-profit scenario) was the most reasonable approach. 

\subsection*{Other Comments}

\begin{itemize}
	\item 4. \textbf{If this paper wants to be more ambitious, it’s important to more clearly differentiate the anal- ysis from what we’d get from simple aggregate computations—I already recommend this in comment 1, but with the paper’s current assumptions the scope is inherently limited. One natural direction to pursue would be incorporating capital heterogeneity (e.g. computers/telecom and houses are very different capital goods, and this has some important implications).}

	Another referee had a similar proposal, which was to look more carefully at the breakdown in capital types. Section 6.2 of the paper now explicitly shows results of the elasticity with respect to structures, with respect to equipment, and with respect to intellectual property. The general outcome is as follows. The structures elasticity is bounded by 0.075-0.150 throughout the period 1948-2018, without a noticeable drift up (in the bounds). The equipment elasticity is bounded by 0.075-0.012 throughout, and again doesn't appear to have an upward drift. Finally, the IP elasticity is around 0.02 in 1948 and this rises to a range of 0.05-0.10 by 2018. This provides more evidence that the bounds on the aggregate elasticity drifted upwards due almost exclusively because of the increased importance of IP. 

	\item 5. \textbf{The lower bound is justified in a slightly strange way: “we know that industries experienced the depreciation of existing capital and their capital costs are at least this large”. Of course (as the paper acknowledges in the next paragraph) this need not be true if the required real rate of return is less than 0. A more accurate justification, I think, would be to say that r (less than) 0 is simply implausible for capital - one could allude to expected returns in public or private equity markets (which are probably related to the return demanded on capital) as evidence}

	Fair point. The text in the paper was updated to be more clear on this (another referee had a similar comment). 

	\item 6. \textbf{I think that there is a typo at the top of page 18: do you mean “0.023 using the user cost assumption” rather than “0.23”? Also, why report just the average difference rather than the average absolute difference (or RMSE, etc.)?}

	Yes, that was a typo. I've played around with a few other summary statistics, and they tell a similar story. I've added some additional information in a root squared difference to the current draft.

\end{itemize}





\end{document}