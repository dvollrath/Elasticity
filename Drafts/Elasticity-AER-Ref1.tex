\documentclass[11pt]{article}
%%%%%%%%%%%%%%%%%%%%%%%%%%%%%%%%%%%%%%%%
\usepackage{amsmath}
\usepackage{verbatim}
\usepackage[usenames,dvipsnames]{color}
\usepackage{setspace}
\usepackage{lscape}
\usepackage{longtable}
\usepackage[top=1.25in,bottom=1.25in,left=1in,right=1in]{geometry}
\usepackage{graphicx}
\usepackage{epstopdf}
\usepackage{epsfig}
\usepackage{fancyhdr}
\usepackage{booktabs}
\usepackage{dcolumn}
\usepackage{arydshln}
\usepackage{natbib}
\usepackage{tabularx}
\usepackage{subfigure}
\usepackage{hyperref}
\usepackage{xcolor}

\hypersetup{
    colorlinks,
    linkcolor={red!50!black},
    citecolor={blue!50!black},
    urlcolor={blue!80!black}
}

\newtheorem{proposition}{Proposition}
\newtheorem{corollary}{Corollary}

\renewcommand{\thetable}{A.\arabic{table}}
\renewcommand{\thesection}{A.\arabic{section}}
\renewcommand{\theequation}{A.\arabic{equation}}
\makeatletter
\renewcommand{\l@section}{\@dottedtocline{1}{1.5em}{2.6em}}
\renewcommand{\l@subsection}{\@dottedtocline{2}{4.0em}{3.6em}}
\renewcommand{\l@subsubsection}{\@dottedtocline{3}{7.4em}{4.5em}}
\makeatother

\setcounter{MaxMatrixCols}{10}
\newcolumntype{d}[1]{D{.}{.}{-2.#1}}
\newenvironment{proof}[1][Proof]{\noindent\textbf{#1.} }{\ \rule{0.5em}{0.5em}}
\setlength{\columnsep}{.2in}
%\psset{unit=1cm}
\newcolumntype{R}{>{\raggedleft\arraybackslash}X}

\def\sym#1{\ifmmode^{#1}\else\(^{#1}\)\fi}

\begin{document}
\noindent {\Large \textbf{\strut The Elasticity of Aggregate Output with Respect to Capital and Labor}}


\section*{Response to Referee 1}
\onehalfspacing 

Thanks to you for reading the paper and offering these comments and questions. I've made several changes to the paper in response. In each case, quotations from your original report are in bold, followed by my response. 

\begin{itemize}
	\item 1. \textbf{Depreciation is not reported by firms, as I understand it, but rather is imputed by national accountants. As such, I found that treatment, even as a lower bound, as somewhat unhelpful, or at least less helpful than the other treatments. The author acknowledges this when the “golden rule” method is applied, but this comes after discussion of the depreciation results. At minimum, the author might highlight more how depreciation is calculated, what are the implied depreciation rates underlying the calculation, and might highlight if, when the underlying asset-specific depreciation rates used change (do they change?), whether that shows up visually in his time series.}

	Point well taken. In Section 4.2.2 I discuss more that depreciation, being imputed, is a ``fuzzy'' boundary. And your point on possible changes in the calculation of deprecation (schedules, etc..) was addressed in a new section in the Appendix, which reviewed available BEA information on this. From that research, I found no demonstrable change in methodology that would explain changes in the depreciation lower bound. 

	\item 2. \textbf{The paper largely ignores potential mismeasurement of the kind suggested in Smith et al. or the possibility of missing investment and capital (which would imply missing GDP), even though it does de-capitalize IP. I personally find both of those possibilities to more plausibly alter our assessment of the elasticity than several of the exercises the author pursues. For my taste, it’d be worth spending more time here.}

	\item 3. \textbf{I found the time spent references BF to be somewhat overkill – particularly (as the author makes explicit and points out himself) for the no profits case. If the author believes that doing the calculations at the industry level rather than at the aggregate matters significantly, he should emphasize it more. The one area where I felt the opposite and wanted more information was on how one thinks about or treats imports and exports in the (Lambda) matrix.}

	I see your point. I've kept this theory section intact for the moment as I have had readers comment that they found it useful for making the following results clear. Should I be fortunate to have this paper move forward, I'd be open to shifting this material (in part or in whole) to an Appendix, should you and the Editor feel that was appropriate. 

	Your point regarding imports and exports is a good one, and echoes a comment by another referee. I've added some material to the paper in Section 3 (on the source input/output data) and the Appendix which addresses how the calculations are impacted by the presence of imports and exports.

	\item 4. \textbf{I know government was removed from some exercises, but would have benefitted from more discussion of its treatment when included. In national accounting, I thought the contribution of capital to government value added was measured by imputing a user cost equal to the depreciation rate (i.e. with the opportunity cost term set to zero). Perhaps I missed it, but in the user-cost measure treatment, is something different done for government? Isn’t that problematic? I’m shaky myself on all this so imagine what was done was correct, but these issues should be more clearly highlighted for the reader.}

	The government industries do present a conceptual question regarding how capital costs are calculated. You are correct that the BEA imputes government value-added by combining labor costs with depreciation of government capital. What that means in my calculations is that the ``no-profits'' capital costs and ``depreciation only'' capital costs are identical for the government. 

	In a sense, one might presume that the government industries should always be treated as ``no-profit'' regardless of how capital costs are calculated in private industries. In an earlier iteration of the paper I considered this, setting capital costs in government equal to value-added minus labor costs (which means capital costs were equal to depreciation) regardless of how capital costs were set in other industries. There was no appreciable difference in the results. Based on that, I stuck with treating the government like other industries just for consistency. But to your point, I didn't mention this explicitly in the paper.

	Maybe the most appreciable difference this should create is with the user cost calculation. In the Appendix A.8 I go through the details of the user cost calculation. In short, I presume that the government has to finance capital like a firm, but the interest rate it faces is the 10-year T-bond rate (so generally lower than in other industries). Again, this does not appear to create any significant difference in the results, but I've made this more clear in the text now.

	\item 5. \textbf{I thought the author could go further than simply offer four treatments and use them to suggest bounds. Most readers, for example, will have priors on what a reasonable distribution of industry-level profit shares would look like. Perhaps the author can report the implied profit share accompanying each of his estimates.}

	A great point, and shared with other referees. I've added a new section to the paper - Section 5.3 - that explicitly discusses the implied aggregate gross output markup (Figure 5) and the implied profit share of value-added under each scenario. Also, Appendix A.7 shows further information on those profit shares and value-added markups. These give a sense of the bounds on those profits shares and/or markups that are consistent with the elasticities. Outside estimates of markups or profit shares all appear to fall within the bounds, and I take this as supportive evidence that the boundaries on the elasticities are reasonable for whatever priors on profits shares/markups that a reader may have. 

	\item 6. \textbf{The author’s main “so what” and applications, as I took it, was that the capital elasticity is likely rising over time and this caries implications for the average rate of TFP growth. Can the author do more to expose what is at stake? Surely the results have some implication for the debate about rapidly rising markups, as in the work by Jan Eckhout and co-authors? Or for tax policy? Helping put the scale of the estimates in those sorts of contexts might be useful.}

	Point taken. In this sense, I think the implication of this is as follows. When markups rise (as in Eckhout and other's work) this implies that TFP growth moves from the lower bound (when markups are low) towards the upper bound (when markups are high). This implies that there is a less pronounced slowdown in TFP growth over the same period, and hence the presumed slowdown is less severe. I've added some additional discussion of this in Section 7.

\end{itemize}



\end{itemize}





\end{document}