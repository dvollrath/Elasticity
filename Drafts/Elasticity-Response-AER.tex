\documentclass[11pt]{article}
%%%%%%%%%%%%%%%%%%%%%%%%%%%%%%%%%%%%%%%%
\usepackage{amsmath}
\usepackage{verbatim}
\usepackage[usenames,dvipsnames]{color}
\usepackage{setspace}
\usepackage{lscape}
\usepackage{longtable}
\usepackage[top=1.25in,bottom=1.25in,left=1in,right=1in]{geometry}
\usepackage{graphicx}
\usepackage{epstopdf}
\usepackage{epsfig}
\usepackage{fancyhdr}
\usepackage{booktabs}
\usepackage{dcolumn}
\usepackage{arydshln}
\usepackage{natbib}
\usepackage{tabularx}
\usepackage{subfigure}
\usepackage{hyperref}
\usepackage{xcolor}

\hypersetup{
    colorlinks,
    linkcolor={red!50!black},
    citecolor={blue!50!black},
    urlcolor={blue!80!black}
}

\newtheorem{proposition}{Proposition}
\newtheorem{corollary}{Corollary}

\renewcommand{\thetable}{A.\arabic{table}}
\renewcommand{\thesection}{A.\arabic{section}}
\renewcommand{\theequation}{A.\arabic{equation}}
\makeatletter
\renewcommand{\l@section}{\@dottedtocline{1}{1.5em}{2.6em}}
\renewcommand{\l@subsection}{\@dottedtocline{2}{4.0em}{3.6em}}
\renewcommand{\l@subsubsection}{\@dottedtocline{3}{7.4em}{4.5em}}
\makeatother

\setcounter{MaxMatrixCols}{10}
\newcolumntype{d}[1]{D{.}{.}{-2.#1}}
\newenvironment{proof}[1][Proof]{\noindent\textbf{#1.} }{\ \rule{0.5em}{0.5em}}
\setlength{\columnsep}{.2in}
%\psset{unit=1cm}
\newcolumntype{R}{>{\raggedleft\arraybackslash}X}

\def\sym#1{\ifmmode^{#1}\else\(^{#1}\)\fi}

\begin{document}
\begin{titlepage}
\vspace{2in} \noindent {\large \today}

\vspace{.5in} \noindent {\Large \textbf{\strut RESPONSES: The Elasticity of Aggregate Output with Respect to Capital and Labor}}

\vspace{.25in} \noindent {\large Dietrich Vollrath}

\vspace{.05in} \noindent University of Houston

\vfill \noindent \textsc{Abstract} \hrulefill

\vspace{.05in} \noindent This contains responses to comments from referees at the AER on an initial submission of this paper.
 
\vspace{.1in} \hrule

\vspace{.1in} \noindent {\small Contact information: 201C McElhinney Hall, U. of Houston, Houston, TX 77204, devollrath@uh.edu.}
\end{titlepage}

\pagebreak 

\section{Overview}
\onehalfspacing I received three referee reports from the AER, and have made changes to the original paper in response to those. This response describes how the current draft has been updated compared to the prior draft, and in some cases why a change suggested was not made. The organization of this response is to first discuss ``large'' changes made in response to comments by any referee, and then it follows up with responses to minor comments by all referees. A response to a single referee is thus spread throughout this document.

There were a few major changes made to the original draft:
\begin{enumerate}
	\item I use BEA Use Tables for the input/output data. This creates a theoretical issue because Use Tables are not defined as industry-by-industry (they are industry-by-commodity). The current draft explains in the main text the issue this creates, and why in practice the numerical effect of this is very small. The Appendix includes a thorough breakdown of the theoretical issue and why it generates such a small numerical effect.
	\item The original draft included estimates of the capital elasticity for a set of OECD countries, and compared those to the US. I have removed those from this draft. Given some referee comments and my own investigation, some of the measurement issues between OECD countries and the US are substantial, and I'm not confident that the elasticities are really ``apples-to-apples'' comparisons. OECD estimates are being pursused in a separate project.
	\item 
\end{enumerate}

\section{Discrepancies in estimates}
The most substantive issue with the paper was raised by referee 2. I have not pasted the entire text of their comment here (it goes on for about two pages) but have selected what I believe are the most relevant passages.

The core issue was that theoretically an estimate using the Baqee/Farhi method under the no-profit assumption should yield elasticities for factors that are identical to the ratio of factor costs to value-added for each factor. In the terminology used in the paper, it should be the case that (for capital) $\epsilon_{Kt} = s_{Kt}^{VA}$ under the no-profit scenario. In the original paper I described a discrepancy in my estimates such that $\epsilon_{Kt} \neq s_{Kt}^{VA}$ under the no-profit scenario, and attempted to explain that, in part, due to a calculation issue. 

The referee - rightly - did not find that explanation correct. The discrepancy had two sources, as it turns out. The first was a classic ``garbage in/garbage out'' coding issue, where the incorrect vector of final-use shares was being used for some years. That first source had helped mask the second, which was my use of the BEA Use Tables for input/output information. Use Tables report the purchases by industry $i$ of commodity $j$, and commodity $j$ is not identical to the output of industry $j$. Because of this conceptual difference between industries and commodities, the gross output of industry $i$ is not identical to the gross output of commodity $j$. In addition, the Use Table includes two commodities (Scrap/Used materials and Noncomparable imports) that have no equivalent industry.

These conceptual issues with the Use Table generate the small discrepancies I found between my estimates of $\epsilon_{Kt}$ and the shares $s_{Kt}^{VA}$ under the no-profit assumption. To be clear, there are also discrepancies in my estimates of $\epsilon_{Kt}$ under other assumptions about capital costs (e.g. depreciation costs, etc.) but there were not independent ratios to compare them to. 

What have I done to address this in the current draft? I am continuing to use the Use Tables, despite the conceptual issue. That is driven by the fact that the Use Tables are available annually from 1948-2018, while other more theoretically valid tables (e.g. Total Requirements Tables that are explicitly industry-by-industry) are only available for shorter time frames. I wanted to have the longest possible time series of estimates. 

To justify that choice, I have included an analysis of the discrepancy or bias introduced by using the Use Tables in Appendix Section A.4. What I demonstrate there is that despite the issues with the Use Table, the bias is only on the order of 0.003-0.005. There are two ways to see this. First, for the elasticity estimates under the no-profits assumption, one can compare $\epsilon_{Kt}$ to $s_{Kt}^{VA}$ directly, and that difference is never more than 0.005. Second, for 1997-2018 I create alternative estimates of $\epsilon_{Kt}$ under all capital cost assumptions based on the BEA's Total Requirements Tables, which do not have the same issues. I then compare those estimates to my baseline using the Use Table, and again the difference is only 0.003-0.005 each year. Despite the conceptual issue with the Use Table, in practice it creates a very small numerical bias in my elasticity estimates. 

Appendix A.4 includes a full theoretical breakdown of the issues in the Use Table, and how it leads to this bias. It comes down to the fact that the vector of gross output of industries, $X_I$, is not identical to the vector of gross output of commodities, $X_C$. As it turns out, those vectors are quite close despite the definitional issue, so $X_C \approx X_I$, and hence the bias is small. The BEA itself implicitly recognizes that these vectors are similar, in that they name commodities using the same exact NAICS codes as industries. 

In the end, then, the referee was correct to push on this point. And they were correct that it was the treatment of the ``extra'' commodities Scrap/Used and Noncomparable imports, as well as the nature of the Use Table, that was creating a theoretical issue. In practice it turns out that the bias from this is small.

\section{Partial Elasticities}
Referee 3 makes a conceptual point: ``\textbf{The author motivates the results as estimating $d \ln Y/d \ln K$, however I don't think this is precisely the object being estimated. I think the precise object must be $\partial \ln Y/ \partial \ln K$ where the partial derivative is holding fixed the allocation of resources in the economy. This is because, when the economy is inefficient, a change in the stock of capital may trigger some endogenous reallocations which will also affect aggregate output.}''

I understand the concern of the referee: what exactly is being held constant in the Baqaee/Farhi (BF) setting when we think about a change in the aggregate capital stock? 

Holding industry-level markups and industry-level total factor productivity constant, 

To the referee's point, this influences the interpretation/use of these elasticities. 

\section{Minor comments}
\subsection{Referee 1}

\subsection{Referee 2}

\subsection{Referee 3}
\begin{itemize}
	\item Comment 2: ``\textbf{In equation (9) the vector Gamma is defined as referring to value-added shares...}''. The referee is correct that this is a typo, a left-over from prior draft. The text has been edited to refer to final-use shares, and as an aside the code doing the calculations was using the correct final-use shares regardless. 
	\item Comment 3: ``\textbf{Housing in corporate sector in Europe...}''. The referee notes that housing is accounted for in national acounts in a different manner than in the US, and therefore the elasticities might not be comparable. This is true, and having investigated further, is one of the reasons that I removed estimates for the OECD from the current draft. Doing consistent estimates across OECD countries (including the US) is now being pursused as a separate project, given the scope of the work.
\end{itemize}
\end{document}