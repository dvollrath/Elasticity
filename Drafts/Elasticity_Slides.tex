%-------------------------------------STYLES----------------------------------------------------------------%

\documentclass[10pt,xcolor=dvipsnames]{beamer}

%\mode<presentation>
%{
\usetheme{Madrid}
%\setbeamercovered{transparent}
%} 
\usefonttheme{professionalfonts}
\usecolortheme{dolphin}

\usepackage{setspace}
\usepackage[english]{babel}
\usepackage[latin1]{inputenc}
\usepackage{times}
\usepackage[T1]{fontenc}
\usepackage{color}
\usepackage{graphicx}
\usepackage{amssymb}
\usepackage{amsthm}
\usepackage{bm}
\usepackage{rotating}
\usepackage{ccaption}
\usepackage{booktabs}
\usepackage{lscape}
\usepackage{colortbl}
\usepackage{arydshln}
\usepackage{tabularx}
\usepackage{appendixnumberbeamer}

%\useinnertheme{rounded}
\setbeamertemplate{items}[balls]
%\usepackage[notocbib]{apacite}                % This is bibliography package.
%\renewcommand{\bibliographytypesize}{\tiny}
\setbeamertemplate{navigation symbols}{}
\usepackage{graphics}
\usepackage{epstopdf}
\DeclareGraphicsRule{.tif}{png}{.png}{`convert #1 `dirname #1`/`basename #1 .tif`.png}
%\usepackage{fleqn}
%\setlength{\mathindent}{1cm}
%\color{white}
%\hypersetup{colorlinks=true,linkcolor=Blue}
\usepackage{bbm}
\newcommand{\backupbegin}{
   \newcounter{framenumberappendix}
   \setcounter{framenumberappendix}{\value{framenumber}}
}
\newcommand{\backupend}{
   \addtocounter{framenumberappendix}{-\value{framenumber}}
   \addtocounter{framenumber}{\value{framenumberappendix}}
}

%% Change the margins
\newenvironment{changemargin}[2]{%
  \begin{list}{}{%
    \setlength{\topsep}{0pt}%
    \setlength{\leftmargin}{#1}%
    \setlength{\rightmargin}{#2}%
    \setlength{\listparindent}{\parindent}%
    \setlength{\itemindent}{\parindent}%
    \setlength{\parsep}{\parskip}%
  }%
  \item[]}{\end{list}}

\newtheorem{proposition}{Proposition}

%-------------------------------------------Title---------------------------------------------------%


\title[Aggregate elasticities]{The Elasticity of Aggregate Output \\ with Respect to Labor and Capital}

\author[Vollrath]{Dietrich Vollrath}
\institute[UH]{University of Houston}

\date[January 2024]{}

%-------------------------------------------Slides---------------------------------------------------%

\begin{document}
\maketitle

\section{Introduction}

\begin{frame}{Research questions}\label{define}

Questions:
\begin{itemize}
  \item What are the elasticities of GDP with respect to capital and labor?
  \item Have those elasticities changed over time?
\end{itemize}

\pause
\vspace{.25in}
Our ``rule of thumb'' is that capital elasticity is $\alpha = 1/3$:
\begin{itemize}
  \item Presumes a coherent aggregate production function
  \item Is based on labor's share of total GDP; presumes zero profits
\end{itemize}

\end{frame}


\begin{frame}{Why do we care?}

The answer informs us on:
\begin{itemize}
  \item Relative importance of labor and capital as factors
  \item Convergence speed, transition dynamics
  \item Distribution of GDP to labor, capital, profits
  \item TFP growth rate
  \item Macro calibrations
\end{itemize}

\end{frame}


\begin{frame}{This paper}

Estimate elasticities with looser assumptions:
\begin{itemize}
  \item Industry-specific elasticities linked through input/output network
  \item Allow for arbitrary market power at industry level
  \item Bound the estimates given issues in measuring capital costs
  \item Applies methodology in Baqaee and Farhi (2017, 2018)
\end{itemize}

\pause
\vspace{.25in}
Scope of work:
\begin{itemize}
  \item Provide estimates for US 1948-2018
  \item Evaluate influence of industry and capital types (e.g. IP, housing)
  \item Re-assess estimates of TFP growth 1948-2018
\end{itemize}

\end{frame}


\begin{frame}{Preview of results}

Robert Solow was kind-of, sort-of right?

\begin{center}
\includegraphics[width=\textwidth]{fig_cap_base_comparison.eps}
\end{center}

\end{frame}

\begin{frame}{Why do we care?}

The answer informs us on:
\begin{itemize}
  \item Relative importance of labor and capital as factors $\rightarrow$ labor matters more?
  \item Convergence speed, transition dynamics $\rightarrow$ shocks dissipate faster?
  \item Distribution of GDP to labor, capital, profits $\rightarrow$ profits lowered labor share?
  \item TFP growth rate $\rightarrow$ higher, but bigger swings in 1990s/2000s
  \item Macro papers calibrated using $\alpha = 1/3$ $\rightarrow$ kinda right?
\end{itemize}

\end{frame}

\begin{frame}{Relevance and contribution}

Literature on labor's share of GDP: Gollin (2002); Young and Zuleta (2013a,b); Elsby, Hobijn, and Sahin (2013); Karabarbounis and Neiman (2014); Gomme and Rupert (2014); Rognlie (2015); Barkai (2017); Smith, Yagan, Zidar, Zwick (2017); Karabarbounis and Neiman (2018); Koh, Santaeulalia-Llopis, Zheng (2018)

\vspace{.25in}
Differences and similarities:
\begin{itemize}
  \item Elasticities don't equal shares if markups $>1$
  \item Elasticities could provide part of explanation for labor share decline
  \item Elasticity calculation explicitly at industry level vs. aggregate
  \item Same data and imputation problems
\end{itemize}

\end{frame}


\begin{frame}{Theoretical Roadmap}

Capital elasticity (same logic for labor elasticity) is $\epsilon_K$:
\begin{itemize}
  \item $\epsilon_K$ is weighted sum of industry-level capital elasticities
  \item Industry-level capital elasticities are inferred from capital cost shares (industry-level cost minimization, Shepherd's lemma)
  \item Weights reflect both industry's share of final use and their share of \textit{costs} in other industries (I/O relationships)
\end{itemize}
\end{frame}

\begin{frame}{Theoretical setting}
\textbf{Borrowed from Baqaee and Farhi (2017, 2018)}
\vspace{.25in}

Each industry $i$ has constant-returns cost function. Industry $i$ has costs as follows:

\begin{equation}
  COST_i = COST_{iM} + COST_{iK} + COST_{iL}
\end{equation}

The first term is total intermediate costs from $J$ total industries:

\begin{equation}
  COST_{iM} = \sum_{j=1}^{J} COST_{ij}
\end{equation}

\end{frame}

\begin{frame}{Theoretical setting}
Cost shares for intermediates defined as

\begin{equation}
  \lambda_{ij} = \frac{COST_{ij}}{COST_i}
\end{equation}

and for factors of production as 

\begin{eqnarray}
  \lambda_{iK} &=& \frac{COST_{iK}}{COST_i} \\
  \lambda_{iL} &=& \frac{COST_{iL}}{COST_i}.
\end{eqnarray}

\end{frame}

\begin{frame}{Theoretical setting}
Build a matrix of intermediate cost shares

\begin{equation}
  \Lambda = 
  \begin{bmatrix}
    \lambda_{11} & \lambda_{12} & \cdots & \lambda_{1J} \\
    \lambda_{21} & \lambda_{22} & \cdots & \lambda_{2J} \\
    \vdots       & \vdots       & \ddots & \vdots       \\
    \lambda_{J1} & \lambda_{J2} & \cdots & \lambda_{JJ}
  \end{bmatrix}
\end{equation}

and a matrix of factor cost shares

\begin{equation}
  B = 
  \begin{bmatrix}
    \lambda_{1K} & \lambda_{1L} \\
    \lambda_{2K} & \lambda_{2L} \\
    \vdots       & \vdots       \\
    \lambda_{JK} & \lambda_{JL}
  \end{bmatrix}
\end{equation}

\end{frame}

\begin{frame}{Theoretical setting}
Final use shares of GDP. Let $GDP = \sum_{j=1}^J F_j$, where $F_j$ is final use of $j$, then

\begin{equation}
  \gamma_j = \frac{F_j}{GDP}.
\end{equation}

Collect in a vector,
\begin{equation}
  \Gamma' = 
  \begin{bmatrix}
    \gamma_1 &
    \gamma_2 &
    \cdots &
    \gamma_J &
  \end{bmatrix}
\end{equation}

\end{frame}

\begin{frame}{Theoretical setting}
Baqaee and Farhi show that elasticities with respect to capital ($\epsilon_K$) and labor ($\epsilon_L$) are as follows:

\begin{equation}
  \begin{bmatrix}
    \epsilon_K &
    \epsilon_L
  \end{bmatrix}
  =
  \Gamma' (I - \Lambda)^{-1} B
\end{equation}

\end{frame}

\begin{frame}{Theoretical intuition}
Look at capital elasticity to break down:

\begin{equation}
  \epsilon_K = \sum_{j}^{J} \left(\sum_{i}^{J} \gamma_i \ell_{ij} \right) \lambda_{jK}
\end{equation}

\begin{itemize}
  \item $\gamma_i $: dollars of $i$ in final use
  \item $\ell_{ij}$: from Leontief inverse $(I - \Lambda)^{-1}$, cost to $j$ of producing one dollar of $i$ given all I/O relationships
  \item $\lambda_{jK}$: share of costs of $j$ that are used on capital (capital elasticity for $j$)  
\end{itemize}

\vspace{.25in}

The weight on each $\lambda_{jK}$ is cost to $j$ of producing all final use

\end{frame}

\begin{frame}{Theoretical setting}
This nests your favorite methods for estimating $\epsilon_K$ and $\epsilon_L$.
\vspace{.25in}

\begin{itemize}
  \item Solow (1957): $\lambda_{ij} = 0$ (no I/O) and $\lambda_{iK} = RK_i/VA_i$ (zero profits)
  \item Hulten (1978): $\lambda_{ij} \neq 0$ (I/O) and $\lambda_{iK} = RK_i/VA_i$ (zero profits)
  \item Hall (1988, 1990): $\lambda_{ij} = 0$ (no I/O) and $\lambda_{iK} = RK/(RK+wL)$ (profits)
\end{itemize}

\end{frame}


\begin{frame}{Implementation}
Big issues in plugging data into the Baqaee and Farhi structure given national accounts. Simplifying

\begin{equation}
  VA = COMP + TAX + PROP + ROS
\end{equation}

Cannot cleanly extract labor or capital costs for any industry
\begin{enumerate}
  \item Proprietors income ($PROP$) contains labor costs, capital costs, economic profits
  \item Residual operating surplus ($ROS$) contains capital costs and economic profits
\end{enumerate}

Problem measuring costs ....

\end{frame}

\begin{frame}{Implementation}\label{Smatch}
....and the industry definitions are not consistent over time.

\vspace{.25in}

\begin{tabular}{lrrrr}
Series  & I/O table  & National accounts & Capital Stock \\ \hline
1947-62 & NAICS 2012 (47 ind) & SIC 1972 & BEA/NAICS 2012 \\
1963-86 & NAICS 2012 (65 ind) & SIC 1972 & BEA/NAICS 2012 \\
1987-96 & NAICS 2012 (65 ind) & SIC 1987 & BEA/NAICS 2012 \\
1997-18 & NAICS 2012 (71 ind) & NAICS 2012 & BEA/NAICS 2012 \\ \hline
\end{tabular}

\vspace{.25in} All data is from the BEA. Used BEA crosswalks and own assumptions to map all data into NAICS 2012 coding that matched the I/O tables.

\vspace{.25in} \hfill \hyperlink{Amatch}{\beamerbutton{Matching}}
\end{frame}

\begin{frame}{Implementation}\label{Smakeuse}
Absence of precise cost information for capital and labor. Strategy:

\begin{itemize}
  \item I/O table reports actual costs of intermediates (sort of)
  \item Allocate proprietors income to calculate labor costs
  \item Construct different series of $\epsilon_K$ and $\epsilon_L$ based on capital cost assumptions
  \item Try to \textit{bound} the elasticities based on theory/data
  \item Undertake variations on assumptions
\end{itemize}
\vspace{.25in} \hfill \hyperlink{Amakeuse}{\beamerbutton{I/O costs}}
\end{frame}

\begin{frame}{Implementation}
\textbf{Labor costs}: Allocate a portion of proprietors income to labor. General principle:

\begin{equation}
  COST_{iLt} = COMP_{it} + PROP_{it}\left(\frac{COMP_{it}}{VA_{it}-PROP_{it}}\right).
\end{equation}

This follows Gomme and Rupert (2004).

\end{frame}

\begin{frame}{No-profit upper bound}
\textbf{Capital costs}: Set the \textit{upper bound} for capital costs by assuming there are zero profits.

\begin{equation}
  COST_{iKt}^{NoProf} = VA_i - COST_{iLt}.
\end{equation}

Gives an \textit{upper bound} for $\epsilon_K$. Note this will be the \textit{lower bound} for $\epsilon_L$.

\end{frame}

\begin{frame}{No-profit upper bound}
\begin{center}
\includegraphics[width=\textwidth]{fig_cap_base_comparison.eps}
\end{center}
\end{frame}

\begin{frame}{Depreciation lower bound}
\textbf{Capital costs}: Set the \textit{lower bound} for capital costs by using the cost of depreciation ($DEPR_{it}$), which is reported by industry. Assumes zero financing costs of existing capital stock.

\begin{equation}
  COST_{iKt}^{Depr} = DEPR_{it}.
\end{equation}

Gives an \textit{lower bound} for $\epsilon_K$. Note this will be the \textit{upper bound} for $\epsilon_L$.

\end{frame}

\begin{frame}{Depreciation lower bound}
\begin{center}
\includegraphics[width=\textwidth]{fig_cap_base_comparison.eps}
\end{center}
\end{frame}

\begin{frame}{Alternative estimates}
\textbf{Capital costs}: Total investment ($INV_{it}$) is reported by industry. Combines replacement of depreciation and purchase of new capital goods. In Golden rule world $INV = RK$. 

\begin{equation}
  COST_{iKt}^{Inv} = INV_{it}.
\end{equation}

Calculate $\epsilon_K$ and $\epsilon_L$. 

\end{frame}

\begin{frame}{Alternative estimates}
\begin{center}
\includegraphics[width=\textwidth]{fig_cap_all_comparison.eps}
\end{center}
\end{frame}

\begin{frame}{Alternative estimates}\label{Susercost}
\textbf{Capital costs}: Calculate the user cost of capital by industry. Three types of capital (structure, equipment, IP).

\begin{equation}
  COST^{User}_{iKt} = \sum_{j \in st,eq,ip} K_{ijt} R_{ijt}.
\end{equation}

where 

\begin{equation}
  R_{ijt} = (Int_{it} - E[\pi_{ijt}] + \delta_{ijt})\frac{1-z_{jt} \tau_t}{1-\tau_t}
\end{equation}

is the rental rate of each type. 

\begin{itemize}
  \item $Int_{it}$: nominal interest rate facing industry $i$
  \item $E[\pi_{ijt}]$: expected inflation of capital type $j$ for industry $i$
  \item $\delta_{ijt}$: depreciation of capital type $j$ for industry $i$ (BEA)
  \item $z_{jt}$: depreciation allowance for capital type $j$ in tax code (BEA)
  \item $\tau_t$: effective corporate tax rate (BEA)
\end{itemize}
\vspace{.25in} \hfill \hyperlink{Ausercost}{\beamerbutton{Details}}
\end{frame}

\begin{frame}{Alternative estimates}
\begin{center}
\includegraphics[width=\textwidth]{fig_cap_all_comparison.eps}
\end{center}
\end{frame}

\begin{frame}{Robustness}
Can alter the general approach in the following ways and get very similar results (i.e. differ in third decimal place):
\begin{itemize}
  \item Use ``After Redefinitions'' I/O tables that re-assign some transactions (1997-2018). 
  \item Excluded imported intermediates from I/O tables (1997-2018). 
  \item Do not allow negative costs for capital (1948-2018). 
  \item Use different assumptions on proprietors income (1948-2018, bigger differences).
\end{itemize}
\end{frame}

\begin{frame}{Aggregate cost shares}
Define the following aggregate cost share for capital:

\begin{equation}
  s^{Cost}_{Kt} = \frac{\sum_{j \in J} COST_{jKt}}{\sum_{j \in J} COST_{jKt} + COST_{jLt}}. \label{EQ_scost}
\end{equation}

If there are zero profits, then $\epsilon_{Kt} \rightarrow s^{Cost}_{Kt}$. 

\vspace{.25in}

If $\epsilon_{Kt} > s^{Cost}_{Kt}$, markups skew costs towards low capital cost industries

\end{frame}

\begin{frame}{Aggregate cost shares}
\begin{center}
\includegraphics[width=\textwidth]{fig_cap_ratio_comparison.eps}
\end{center}
\end{frame}


\begin{frame}{Private business sector}
Private sector business only:
\begin{itemize}
   \item Exclude government (cost shares not far from average)
   \item Exclude housing (relatively high capital and low labor cost)
\end{itemize} 

Lower implied capital elasticity (and higher labor elasticity)

\end{frame}

\begin{frame}{Private business sector}
\begin{center}
\includegraphics[width=\textwidth]{fig_cap_priv_comparison.eps}
\end{center}
\end{frame}


\begin{frame}{IP Adjustment}\label{Sip}
Intellectual property?

\begin{itemize}
  \item Elasticity rises over time
  \item But may be because data on IP from pre-1990 is scarce?
  \item Koh, Santaeulalia-Llopis, Zheng (2018): aggregate labor share falls due to IP accounting
\end{itemize}
\vspace{.25in} \hfill \hyperlink{Aip}{\beamerbutton{Details}}
\end{frame}

\begin{frame}{IP Adjustment}
\begin{center}
\includegraphics[width=\textwidth]{fig_cap_noip_comparison.eps}
\end{center}
\end{frame}


\begin{frame}{Firm-level data}\label{Sfirm}
De Loecker, Eeckhout, and Unger (2020) use Compustat to infer/estimate cost shares of firms: 

\begin{itemize}
  \item Use firms within an industry $i$ to set industry-level cost shares, $\lambda_{iK}$
  \item Use income statement information or production function estimates
  \item I use their estimates to calculate $\epsilon_K$ and $\epsilon_L$
  \item Not firm-level estimates, estimates consistent with Compustat firm-level data
  \item Compare to private business sector estimates
\end{itemize}

\vspace{.25in} \hfill \hyperlink{Afirm}{\beamerbutton{Details}}
\end{frame}

\begin{frame}{Firm-level data}
\begin{center}
\includegraphics[width=\textwidth]{fig_cap_dleu_comparison.eps}
\end{center}
\end{frame}

\begin{frame}{Market power and the labor share}
Different assumptions about capital costs imply different levels of profits and markups:

\begin{equation}
    \mu^{VA}_{t} = \frac{\sum_{j=1}^{J} VA_{jt}}{\sum_{j=1}^J COST_{jKt} + COST_{jLt}}. \label{EQ_markup}
\end{equation}

What do these assumptions about capital costs imply about markups?

\end{frame}

\begin{frame}{Market power and the labor share}
\begin{center}
\includegraphics[width=\textwidth]{fig_cap_markupva_comparison_dleu.eps}
\end{center}
\end{frame}

\begin{frame}{Market power and the labor share}
Did labor's share of GDP fall because $\epsilon_K$ went up ($\epsilon_L$ went down) or because profits went up?

\begin{itemize}
  \item Bounds on $\epsilon_K$ do appear to rise near end of period
  \item Series based on investment costs, Compustat, show no trend in $\epsilon_K$
  \item Lean towards profit explanation, consistent with markups?
\end{itemize}

\end{frame}

\begin{frame}{Growth Accounting}
For a typical growth accounting exercise:

\begin{equation}
  d \ln TFP_t = d \ln Y_t - \epsilon_{Kt} d \ln K_t - \epsilon_{Lt} d \ln L_t. \label{EQ_accounting}
\end{equation}

\begin{itemize}
  \item The implied growth in TFP depends on the elasticities
  \item BLS uses the ``no-profit'' assumption only
  \item Effect is ambiguous 
\end{itemize}

\end{frame}

\begin{frame}{Growth Accounting}
\begin{center}
\includegraphics[width=\textwidth]{fig_tfp_comparison.eps}
\end{center}
\end{frame}

\begin{frame}{Growth Accounting}
\begin{center}
\begin{tabular}{lccccc}
\midrule
        & \multicolumn{5}{c}{Assumption on capital costs:} \\ \cmidrule(lr){2-6}
        & \multicolumn{3}{c}{National accounts only:} & \multicolumn{2}{c}{Compustat derived:} \\ \cmidrule(lr){2-4} \cmidrule(lr){5-6}
 &  No-profit & Invest. cost & Depr. cost & Prod. fct. & Cost shares\\
Years & (1) & (2) & (3) & (4) & (5) \\
\midrule
\csname @@input\endcsname tab_tfp_scenario.txt
\midrule
\end{tabular}
\end{center}
\end{frame}

\begin{frame}{Conclusions}
Revisiting results:
\begin{itemize}
  \item The rule-of-thumb capital elasticity (1/3) is an upper bound
  \item Realistic assumptions lower estimate (IP, housing, markups)
  \item Profits, not elasticities, explain decline in labor share (?)
  \item TFP growth is higher on average, more dramatic swings in 1990-2018 period
\end{itemize}

\vspace{.25in}
More broadly, labor is ``more important'' than normally assumed
\end{frame}

\section{Appendix}

\begin{frame}{Matching}\label{Amatch}
$ELEM_{it}^{NAICS}$ is some element (e.g. labor compensation, depreciation) that I need for a NAICS industry $i$, but only reported on SIC basis as $ELEM_{jt}^{SIC}$ for industry $j$. General concept is this:

\begin{equation}
  ELEM_{it}^{NAICS} = VALU_{it}^{NAICS} \times \frac{ELEM_{jt}^{SIC}}{VALU_{jt}^{SIC}}. \label{EQ_match_app}
\end{equation}

Still requires linking a NAICS industry to an appropriate SIC industry(s). $VALU_{it}^{NAICS}$ and $VALU_{jt}^{SIC}$ are available in all cases.

\end{frame}

\begin{frame}{Matching}
From easy to hard:
\begin{itemize}
  \item \textbf{One SIC to one NAICS:} 1947-62 the SIC industry ``Construction'' (SIC 1972 code C) is matched to NAICS industry ``Construction'' (NAICS code 23). 
  \begin{itemize}
    \item Apply formula.
  \end{itemize}
  \item \textbf{One SIC to many NAICS:} 1947-62 the SIC industry ``Retail trade'' (SIC code G) is matched to NAICS industries ``Retail trade'' (NAICS code 44RT) and ``Food service and drinking places'' (NAICS code 722). 
  \begin{itemize}
    \item Apply same SIC ratio $ELEM_{jt}^{SIC}/VALU_{jt}^{SIC}$ to multiple NAICS industries. 
  \end{itemize}
\end{itemize}

\end{frame}

\begin{frame}{Matching}
From easy to hard:
\begin{itemize}
  \item \textbf{Many SIC to one NAICS:} 1947-62 ``Banking'' (SIC code 60), ``Credit agencies'' (SIC code 61), ``Security and commodity brokers'' (SIC code 62), ``Insurance carriers'' (SIC code 63), and ``Insurance agents, brokers'' (SIC code 64) all being matched to NAICS industry ``Finance and Insurance'' (NAICS code 52). 
  \begin{itemize}
    \item Sum SIC industry $ELEM_{jt}^{SIC}$ and sum SIC industry $VALU_{jt}^{SIC}$ and then calculate $ELEM_{jt}^{SIC}/VALU_{jt}^{SIC}$, apply to NAICS industry.
  \end{itemize}

\vspace{.25in} \hfill \hyperlink{Smatch}{\beamerbutton{Return}}
\end{itemize}

\end{frame}


\begin{frame}{From Make/Use to Cost Shares}\label{Amakeuse}
BEA I/O tables distinguish $J$ industries from $M$ commodities, although for most practical purposes these align (e.g. agriculture industry produces the agricultural commodity), but not exactly. Commodities like ``used/scrap'' or ``noncomparable imports'' exist. 

\begin{itemize}
  \item \textbf{Use Table:} $U$, is a $M \times J$ matrix. $u_{mj}$ shows the amount of a commodity $m$ used as an input by industry $j$
  \item \textbf{Make Table:} $V$, is a $J \times M$ matrix. $v_{jm}$ shows the amount produced by industry $j$ of commodity $m$
  \item $X_M$ measure the gross output of each of the $M$ commodities ($M$ by 1)
  \item $X_I$ measure the gross output of each of the $J$ industries ($J$ by 1)
  \item $F_I$ measure the final use of each of the $J$ industries ($J$ by 1)
\end{itemize}

\end{frame}

\begin{frame}{From Make/Use to Cost Shares}
Let
\begin{equation}
  A = V \hat{X}_M^{-1}
\end{equation}
$A$ is a $J \times M$ matrix. $a_{im}$ measures the share of gross output of commodity $m$ that is produced by industry $i$. 
\vspace{.25in}
Let
\begin{equation}
  C = A U =  V \hat{X}_M^{-1} U.
\end{equation}
$C$ is a $J \times J$ matrix. $c_{ij}$ is the spending by industry $j$ on output of industry $i$ (working through the commodities used by $j$ and produced by $i$). 

\vspace{.25in}

$C$ is the matrix of costs used in the calculation of the elasticities
\end{frame}

\begin{frame}{From Make/Use to Cost Shares}
Confirm that this logic and calculation is sound. Let:
\begin{eqnarray}
  X_I &=& Ce + F_I \\
  X_I &=& C'e + V_I
\end{eqnarray}
where $e$ is a $Jx1$ vector of 1's, meaning gross output is the sum of intermediate sales and final use sales, or gross output is the sum of intermediate purchases plus value added.

\vspace{.25in}

Given $C$, $F_I$, $X_I$, can solve for $V_I$, industry value-added. Short of small rounding errors this is equal to BEA reported value-added, confirming calculations.

\vspace{.25in} \hfill \hyperlink{Smakeuse}{\beamerbutton{Return}}
\end{frame}

\begin{frame}{Implementation}\label{Ausercost}
...and $Int_{it}$ defined as

\begin{equation}
    Int_{it} = \sum_m s_{imt} Int_{mt}
\end{equation} 

\begin{itemize}
  \item $s_{imt}$ shares of financing from source $m$ (e.g. mortgages, corp bonds). From Fed Flow of Funds.
  \item $Int_{mt}$ are source-specific interest rates (e.g. 30-year mortgage rate, 30-year AAA bond rate). From Fed.
\end{itemize}

\end{frame}


\begin{frame}{Implementation}
...and $E[\pi_{ijt}]$ is
\begin{itemize}
  \item Proxied with average inflation over following three years.
  \item BEA reports industry-specific, $i$, capital-type specific, $j$, price indices
  \item Variations (backward looking, 5-year) give similar results
\end{itemize}

\vspace{.25in} \hfill \hyperlink{Susercost}{\beamerbutton{Return}}
\end{frame}

\begin{frame}{Excluding IP}\label{Aip}
To exclude IP as a capital cost, have to remove own-account IP spending and accumulation:
\begin{itemize}
  \item Remove from value-added: $VALU_{it}^{NoIP} = VALU_{it} - INV_{i,IP,t}$
  \item Remove from investment spending: $INV_{it}^{NoIP} = INV_{it} - INV_{i,IP,t}$
  \item Remove from depreciation: $DEPR_{it}^{IP} = DEPR_{it} - DEPR_{i,IP,t}$
  \item Remove from capital stock: $K^{NoIP}_{it} = K_{it} - K_{i,IP,t}$
\end{itemize}
What this does not account for is IP capital purchased from other industries. So adjustment for IP should be larger. 

\vspace{.25in} \hfill \hyperlink{Aip}{\beamerbutton{Return}}
\end{frame}

\begin{frame}{Compustat}\label{Afirm}
Two types of estimates from De Loecker, Eeckhout, Unger (2020)

\begin{itemize}
  \item \textbf{Cost data:} 
  \begin{itemize}
    \item DLEU calculate capital costs from firm level data. 
    \item DLEU calculate Non-capital costs are COGS and SGA from firm-level data.
    \item I calculate ratio of capital to non-capital costs for firms in given industry. 
    \item Multiply that firm-derived ratio by sum of intermediate and labor costs for NAICS industry to get NAICS industry level capital costs. 
  \end{itemize} 
  \item \textbf{Production function:} 
  \begin{itemize}
    \item DLEU estimate industry-level elasticities w.r.t. capital, COGS, and SGA using firm-level data. 
    \item I calculate ratio of capital elasticity to COGS and SGA elasticities. 
    \item Multiply that ratio by sum of intermediate and labor costs for NAICS industry to get NAICS industry level capital costs.
  \end{itemize}
\end{itemize}

\vspace{.25in} \hfill \hyperlink{Sfirm}{\beamerbutton{Return}}
\end{frame}

\end{document}
