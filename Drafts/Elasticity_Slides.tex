%-------------------------------------STYLES----------------------------------------------------------------%

\documentclass[10pt, xcolor=dvipsnames]{beamer}

%\mode<presentation>
%{
\usetheme{Madrid}
%\setbeamercovered{transparent}
%} 
\usefonttheme{professionalfonts}
\usecolortheme{beaver}

\usepackage{setspace}
\usepackage[english]{babel}
\usepackage[latin1]{inputenc}
\usepackage{times}
\usepackage[T1]{fontenc}
\usepackage{color}
\usepackage{graphicx}
\usepackage{amssymb}
\usepackage{amsthm}
\usepackage{bm}
\usepackage{rotating}
\usepackage{ccaption}
\usepackage{booktabs}
\usepackage{lscape}
\usepackage{colortbl}
\usepackage{arydshln}
\usepackage{tabularx}
\usepackage{appendixnumberbeamer}

%\useinnertheme{rounded}
\setbeamertemplate{items}[balls]
%\usepackage[notocbib]{apacite}                % This is bibliography package.
%\renewcommand{\bibliographytypesize}{\tiny}
\setbeamertemplate{navigation symbols}{}
\usepackage{graphics}
\usepackage{epstopdf}
\DeclareGraphicsRule{.tif}{png}{.png}{`convert #1 `dirname #1`/`basename #1 .tif`.png}
%\usepackage{fleqn}
%\setlength{\mathindent}{1cm}
%\color{white}
%\hypersetup{colorlinks=true,linkcolor=Blue}
\usepackage{bbm}
\newcommand{\backupbegin}{
   \newcounter{framenumberappendix}
   \setcounter{framenumberappendix}{\value{framenumber}}
}
\newcommand{\backupend}{
   \addtocounter{framenumberappendix}{-\value{framenumber}}
   \addtocounter{framenumber}{\value{framenumberappendix}}
}

%% Change the margins
\newenvironment{changemargin}[2]{%
  \begin{list}{}{%
    \setlength{\topsep}{0pt}%
    \setlength{\leftmargin}{#1}%
    \setlength{\rightmargin}{#2}%
    \setlength{\listparindent}{\parindent}%
    \setlength{\itemindent}{\parindent}%
    \setlength{\parsep}{\parskip}%
  }%
  \item[]}{\end{list}}

\newtheorem{proposition}{Proposition}

%-------------------------------------------Title---------------------------------------------------%


\title[Aggregate elasticities]{The Elasticity of Aggregate Output with Respect to Labor and Capital}

\author[Vollrath]{Dietrich Vollrath \inst{1}}
\institute[UH]{\inst{1} University of Houston}

\date[April 2021]{}

%-------------------------------------------Slides---------------------------------------------------%

\begin{document}
\maketitle

\section{Introduction}

\begin{frame}{Research questions}\label{define}

\begin{itemize}
  \item What are the elasticities of GDP with respect to capital and labor?
  \item Have those elasticities changed over time?
  \item Are those elasticities similar across countries? 
\end{itemize}

\end{frame}

\begin{frame}{Summary}

This paper answers those questions:
\begin{itemize}
  \item Calculate elasticities for US 1948-2018, OECD 1995-2015
  \item Applies the methodology of Baqaee and Farhi (2017, 2018)
  \item Industry-specific cost structure, market power, and input-output relationships
  \item Creates bounds for those elasticities given issues with measuring capital costs
\end{itemize}

\end{frame}

\begin{frame}{Preview of U.S. results}

Robert Solow was kind-of, sort-of right?

\begin{center}
\includegraphics[width=0.9\textwidth]{fig_cap_base_comparison.eps}
\end{center}

\end{frame}

\begin{frame}{Relevance and contribution}

The answer informs us on:
\begin{itemize}
  \item Consequences of aggregate shocks
  \item Convergence speed, transition dynamics
  \item Distribution of GDP to labor, capital, profits
  \item Validity of decades of macro papers calibrated using $\alpha = 1/3$
\end{itemize}

\end{frame}

\begin{frame}{Relevance and contribution}

Literature on labor's share of GDP: Gollin (2002); Young and Zuleta (2013a,b); Elsby, Hobijn, and Sahin (2013); Karabarbounis and Neiman (2014); Gomme and Rupert (2014); Rognlie (2015); Barkai (2017); Smith, Yagan, Zidar, Zwick (2017); Karabarbounis and Neiman (2018); Koh, Santaeulalia-Llopis, Zheng (2018)

\vspace{.25in}
Differences and similarities:
\begin{itemize}
  \item Elasticities don't equal shares if markups $>1$
  \item Elasticities could provide part of explanation for labor share decline
  \item Elasticity calculation explicitly at industry level vs. aggregate
  \item Same data and imputation problems
\end{itemize}

\end{frame}




\begin{frame}{Theoretical setting}
\textbf{Borrowed completely from Baqaee and Farhi (2017, 2018)}
\vspace{.25in}

Each industry $i$ has constant-returns cost function. Industry $i$ has costs as follows:

\begin{equation}
  COST_i = COST_{iM} + COST_{iK} + COST_{iL}
\end{equation}

The first term is total intermediate costs from $J$ total industries:

\begin{equation}
  COST_{iM} = \sum_{j=1}^{J} COST_{ij}
\end{equation}

\end{frame}

\begin{frame}{Theoretical setting}
Cost shares for intermediates defined as

\begin{equation}
  \lambda_{ij} = \frac{COST_{ij}}{COST_i}
\end{equation}

and for factors of production as 

\begin{eqnarray}
  \lambda_{iK} &=& \frac{COST_{iK}}{COST_i} \\
  \lambda_{iL} &=& \frac{COST_{iL}}{COST_i}.
\end{eqnarray}

\end{frame}

\begin{frame}{Theoretical setting}
Build the matrix of cost shares

\begin{equation}
  \Lambda = 
  \begin{bmatrix}
    \lambda_{11} & \lambda_{12} & \cdots & \lambda_{1J} & \lambda_{1K} & \lambda_{1L} \\
    \lambda_{21} & \lambda_{22} & \cdots & \lambda_{2J} & \lambda_{2K} & \lambda_{2L} \\
    \vdots       & \vdots       & \ddots & \vdots       & \vdots       & \vdots  \\
    \lambda_{J1} & \lambda_{J2} & \cdots & \lambda_{JJ} & \lambda_{JK} & \lambda_{JL} \\
    0 & 0 & \cdots & 0 & 0& 0 \\
    0 & 0 & \cdots & 0 & 0& 0 \\
  \end{bmatrix}
\end{equation}

where labor and capital are treated as ``industries'' that provide an input to other industries.

\end{frame}


\begin{frame}{Theoretical setting}
Value-added shares of GDP, $VA = \sum_{j=1}^J va_j$, 

\begin{equation}
  \gamma_j = \frac{va_j}{VA}.
\end{equation}

Collect in a vector,
\begin{equation}
  \Gamma' = 
  \begin{bmatrix}
    \gamma_1 &
    \gamma_2 &
    \cdots &
    \gamma_J &
    0 &
    0
  \end{bmatrix}
\end{equation}

\end{frame}

\begin{frame}{Theoretical setting}
Calculate ``cost-based'' Domar weights - value-added weights times the Leontief inverse. 

\begin{equation}
  E = \Gamma' (I - \Lambda)^{-1} \label{EQ_E}
\end{equation}

The structure of $E$ is as follows,

\begin{equation}
  E = 
  \begin{bmatrix}
    \epsilon_1 &
    \epsilon_2 &
    \cdots &
    \epsilon_J &
    \epsilon_K &
    \epsilon_L
  \end{bmatrix}
\end{equation}

\end{frame}


\begin{frame}{Theoretical setting}
Baqaee and Farhi prove that the entries $\epsilon_K$ and $\epsilon_L$ are the elasticity of aggregate output (GDP) with respect to the aggregate stock of $K$ and $L$. 

\begin{itemize}
  \item Domar weights capture elasticity of GDP w.r.t. productivity in industry $i$, capturing downstream and upstream effects
  \item For labor and capital ``industries'', productivity increase is expansion of their supply.
  \item Only downstream effects of labor and capital, as they purchase no inputs.
  \item ``Cost-based'' Domar weights deals with arbitrary markups of prices over costs in industries. 
\end{itemize}

\end{frame}

\begin{frame}{Theoretical setting}
This nests your favorite methods for estimating $\epsilon_K$ and $\epsilon_L$.
\vspace{.25in}

Solow (1957)
\begin{itemize}
  \item No I/O structure: $\lambda_{ij} = 0$ for any two industries
  \item Zero profits: $\lambda_{iK} = RK_i/VA_i$ and $\lambda_{iL} = wL_i/VA_i$
  \item Result is that $\epsilon_K = RK/VA$ and $\epsilon_L = wL/VA$
\end{itemize}

\end{frame}

\begin{frame}{Theoretical setting}

Hall (1988, 1990)
\begin{itemize}
  \item No I/O structure: $\lambda_{ij} = 0$ for any two industries
  \item Non-zero profits: $\lambda_{iK} = RK_i/(RK_i + wL_i)$ and $\lambda_{iL} = wL_i/(RK_i + wL_i)$
  \item Result is that $\epsilon_K = RK/(RK+wL)$ and $\epsilon_L = wL/(RK+wL)$
  \item Aside on markup ($\mu$) and shares of GDP (similar for labor):
    \begin{equation}
      \epsilon_K = \frac{Y}{RK+wL}\frac{RK}{Y} = \mu \frac{RK}{Y}
    \end{equation}
\end{itemize}

\end{frame}

\begin{frame}{Theoretical setting}

Hulten (1978)
\begin{itemize}
  \item I/O structure: $\lambda_{ij} \neq 0$ for any two industries
  \item Zero profits: $\lambda_{iK} = RK_i/VA_i$ and $\lambda_{iL} = wL_i/VA_i$
  \item Result is that $\epsilon_K = RK/VA$ and $\epsilon_L = wL/VA$
  \item Envelope result. No distortions so I/O structure is irrelevant
\end{itemize}

\end{frame}

\begin{frame}{Implementation}
Big issues in plugging data into the Baqaee and Farhi structure given national accounts. Simplifying

\begin{equation}
  GDP = COMP + TAX + PROP + ROS
\end{equation}

Cannot cleanly extract labor or capital costs for any industry
\begin{enumerate}
  \item Proprietors income ($PROP$) contains labor costs, capital costs, economic profits
  \item Residual operating surplus ($ROS$) contains capital costs and economic profits
\end{enumerate}

Problem measuring costs ....

\end{frame}

\begin{frame}{Implementation}
....and the industry definitions are not consistent over time.

\vspace{.25in}

\begin{tabular}{lrrrr}
Series  & I/O table  & National accounts & Capital Stock \\ \hline
1947-62 & NAICS 2012 (47 ind) & SIC 1972 & BEA/NAICS 2012 \\
1963-86 & NAICS 2012 (65 ind) & SIC 1972 & BEA/NAICS 2012 \\
1987-96 & NAICS 2012 (65 ind) & SIC 1987 & BEA/NAICS 2012 \\
1997-18 & NAICS 2012 (71 ind) & NAICS 2012 & BEA/NAICS 2012 \\ \hline
\end{tabular}

\vspace{.25in} All data is from the BEA. Used BEA crosswalks and own assumptions to map all data into NAICS 2012 coding that matched the I/O tables.
\end{frame}

\begin{frame}{Implementation}
Absence of precise cost information for capital and labor. Strategy:

\begin{itemize}
  \item I/O table reports actual costs of intermediates, no problem
  \item Allocate proprietors income to calculate labor costs
  \item Construct different series of $\epsilon_K$ and $\epsilon_L$ based on capital cost assumptions
  \item Try to \textit{bound} the elasticities based on theory/data
  \item Undertake variations on assumptions, do they stay in bounds?
\end{itemize}

\end{frame}

\begin{frame}{Implementation}
\textbf{Labor costs}: Allocate a portion of proprietors income to labor. General principle:

\begin{equation}
  COST_{iLt} = COMP_{it} + PROP_{it}\left(\frac{COMP_{it}}{VA_{it}-PROP_{it}}\right).
\end{equation}

This follows Gomme and Rupert (2004).

\end{frame}

\begin{frame}{Implementation}
\textbf{Capital costs}: Set the \textit{upper bound} for capital costs by assuming there are zero profits.

\begin{equation}
  COST_{iKt}^{NoProf} = VA_i - COST_{iLt}.
\end{equation}

Gives an \textit{upper bound} for $\epsilon_K$. Note this will be the \textit{lower bound} for $\epsilon_L$.

\end{frame}

\begin{frame}{Implementation}
\begin{center}
\includegraphics[width=0.8\textwidth]{fig_cap_1_comparison.eps}
\end{center}
\end{frame}

\begin{frame}{Implementation}
\textbf{Capital costs}: Set the \textit{lower bound} for capital costs by using the cost of depreciation ($DEPR_{it}$), which is reported by industry. Assumes zero financing costs of existing capital stock.

\begin{equation}
  COST_{iKt}^{Depr} = DEPR_{it}.
\end{equation}

Gives an \textit{lower bound} for $\epsilon_K$. Note this will be the \textit{upper bound} for $\epsilon_L$.

\end{frame}

\begin{frame}{Implementation}
\begin{center}
\includegraphics[width=0.8\textwidth]{fig_cap_2_comparison.eps}
\end{center}
\end{frame}

\begin{frame}{Implementation}
\textbf{Capital costs}: Total investment ($INV_{it}$) is reported by industry. Combines replacement of depreciation and purchase of new capital goods. In Golden rule world $INV = RK$. 

\begin{equation}
  COST_{iKt}^{Inv} = INV_{it}.
\end{equation}

Calculate $\epsilon_K$ and $\epsilon_L$. 

\end{frame}

\begin{frame}{Implementation}
\begin{center}
\includegraphics[width=0.8\textwidth]{fig_cap_3_comparison.eps}
\end{center}
\end{frame}

\begin{frame}{Implementation}
\textbf{Capital costs}: Calculate the user cost of capital by industry. Three types of capital (structure, equipment, IP).

\begin{equation}
  COST^{User}_{iKt} = \sum_{j \in st,eq,ip} K_{ijt} R_{ijt}.
\end{equation}

where 

\begin{equation}
  R_{ijt} = (Int_{it} - E[\pi_{ijt}] + \delta_{ijt})\frac{1-z_{jt} \tau_t}{1-\tau_t}
\end{equation}

is the rental rate of each type. 

\begin{itemize}
  \item $Int_{it}$: nominal interest rate facing industry $i$
  \item $E[\pi_{ijt}]$: expected inflation of capital type $j$ for industry $i$
  \item $\delta_{ijt}$: depreciation of capital type $j$ for industry $i$ (BEA)
  \item $z_{jt}$: depreciation allowance for capital type $j$ in tax code (BEA)
  \item $\tau_t$: effective corporate tax rate (BEA)
\end{itemize}

\end{frame}

%\begin{frame}{Implementation}
%...and $Int_{it}$ defined as
%
%\begin{equation}
%    Int_{it} = \sum_m s_{imt} Int_{mt}
%\end{equation} 
%
%\begin{itemize}
%  \item $s_{imt}$ shares of financing from source $m$ (e.g. mortgages, corp bonds). From Fed Flow of Funds.
%  \item $Int_{mt}$ are source-specific interest rates (e.g. 30-year mortgage rate, 30-year AAA bond rate). From Fed.
%\end{itemize}

%\end{frame}


%\begin{frame}{Implementation}
%...and $E[\pi_{ijt}]$ is
%\begin{itemize}
%  \item Proxied with average inflation over following three years.
%  \item BEA reports industry-specific, $i$, capital-type specific, $j$, price indices
%  \item Variations (backward looking, 5-year) give similar results
%\end{itemize}

%\end{frame}

\begin{frame}{Implementation}
\begin{center}
\includegraphics[width=0.8\textwidth]{fig_cap_all_comparison.eps}
\end{center}
\end{frame}


\begin{frame}{Variations}
Private sector business only:
\begin{itemize}
   \item Exclude government (cost shares not far from average)
   \item Exclude housing (relatively high capital and low labor cost)
\end{itemize} 

Lower implied capital elasticity (and higher labor elasticity)

\end{frame}

\begin{frame}{Variations}
\begin{center}
\includegraphics[width=0.9\textwidth]{fig_cap_priv_comparison.eps}
\end{center}
\end{frame}


\begin{frame}{Variations}
Intellectual property?

\begin{itemize}
  \item Elasticity rises over time
  \item But may be because data on IP from pre-1990 is scarce?
  \item Koh, Santaeulalia-Llopis, Zheng (2018): aggregate labor share falls due to IP accounting
\end{itemize}

\end{frame}

\begin{frame}{Variations}
\begin{center}
\includegraphics[width=0.9\textwidth]{fig_cap_ip_comparison.eps}
\end{center}
\end{frame}

\begin{frame}{Aggregate cost shares}
Define the following aggregate cost share for capital:

\begin{equation}
  s^{Cost}_{Kt} = \frac{\sum_{j \in J} COST_{jKt}}{\sum_{j \in J} COST_{jKt} + COST_{jLt}}. \label{EQ_scost}
\end{equation}

If there were no I/O relatonships, then $\epsilon_{Kt} \rightarrow s^{Cost}_{Kt}$. 

\end{frame}

\begin{frame}{Aggregate cost shares}
\begin{center}
\includegraphics[width=0.9\textwidth]{fig_cap_total_comparison.eps}
\end{center}
\end{frame}


\begin{frame}{Decomposition}
From the calculation of $E$,

\begin{equation}
  \epsilon_{Kt} = \sum_{i \in J} va_{it} \ell_{iKt}
\end{equation}

where $va_{it}$ are value-added shares and $\ell_{iKt}$ are Leontief inverse entries. 

\vspace{.25in} Do an Olley-Pakes type decomposition of $\epsilon_{Kt}$

\begin{equation}
  \epsilon_{Kt} = \overline{\ell}_{Kt} + \sum_{i \in J} (va_{it} - \overline{va}_{t})(\ell_{iKt}-\overline{\ell}_{Kt}), \label{EQ_op}
\end{equation}

where $\overline{\ell}_{Kt}$ is mean industry elasticity, and summation is covariance of industry elasticity and size of industry.

\end{frame}

\begin{frame}{Decomposition}
\begin{center}
\includegraphics[width=0.9\textwidth]{fig_cap_op_comparison.eps}
\end{center}
\end{frame}


\begin{frame}{Comparison}
\textbf{OECD}: Create similar bounds for OECD countries using data 2005-2015.

\begin{itemize}
  \item STAN database for I/O accounts and national accounts data by industry
  \item STAN does not separate out proprietors income
  \item 2005-2015: ISIC v.4
  \item Less re-mapping necessary between I/O and national accounts
\end{itemize}

\end{frame}

\begin{frame}{Comparison}
\textbf{Labor costs}: Without proprietors income

\begin{equation}
  COST_{iLt} = COMP_{it} + SELF_{it}\frac{COMP_{it}}{EMPL_{it}}.
\end{equation}

\begin{itemize}
  \item $SELF_{it}$ are self-employed and $EMPL_{it}$ are formal employees. 
  \item Probably understates labor costs as proprietors tend to be high wage.
\end{itemize}

\end{frame}

\begin{frame}{Comparison}
\textbf{Capital costs}: Do similar bounding exercise

\vspace{.25in}
Upper bound on capital elasticity using

\begin{equation}
  COST_{iKt}^{NoProf} = VA_i - COST_{iLt}.
\end{equation}

\vspace{.25in}
Lower bound on capital elasticity using

\begin{equation}
  COST_{iKt}^{Depr} = DEPR_{it}.
\end{equation}

\end{frame}

\begin{frame}{Comparison}
\begin{center}
\includegraphics[width=0.9\textwidth]{fig_cap_oecd_comparison.eps}
\end{center}
\end{frame}

\begin{frame}{Growth Accounting}
For a typical growth accounting exercise:

\begin{equation}
  d \ln TFP_t = d \ln Y_t - \epsilon_{Kt} d \ln K_t - \epsilon_{Lt} d \ln L_t. \label{EQ_accounting}
\end{equation}

\begin{itemize}
  \item The implied growth in TFP depends on the elasticities
  \item BLS uses the ``no-profit'' assumption only
  \item Effect is ambiguous 
\end{itemize}

\end{frame}

\begin{frame}{Growth Accounting}
\begin{center}
\includegraphics[width=0.9\textwidth]{fig_tfp_comparison.eps}
\end{center}
\end{frame}

\begin{frame}{Conclusions}
\begin{itemize}
  \item Naive capital elasticity (1/3) is an upper bound for most of period
  \item ...but this bound shifted up over time
  \item Scope of economic activity matters: housing pulls up capital elasticity
  \item Structural change did not appear to drive shifts, within-industry changes
  \item Consistent across the OECD for short time frame
\end{itemize}
\vspace{.25in}
Remaining work, caveats, and questions...
\begin{itemize}
  \item What are second-order effects for large changes in $K$ or $L$?
  \item Are cost structures consistent with industry-level estimates?
  \item Can infer markups from elasticities and shares. Consistent with firm-level evidence?
  \item Expand OECD coverage over time
\end{itemize}
\end{frame}

\end{document}
